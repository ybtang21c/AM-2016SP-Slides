% !Mode:: "TeX:UTF-8"

\begin{frame}
	\frametitle{第五讲、空间平面与直线}
	\linespread{1.5}
	\begin{enumerate}
	  \item {\bf 内容与要求}{\color{blue}( \S8.2 )}
	  \begin{itemize}
	    \item 熟练掌握空间平面与直线的方程
	    \item 熟练掌握点、直线、平面的位置关系及其表示
% 	    \item 熟练掌握空间平面和直线的方程
% 	    \begin{itemize}
% 	      \item 可分离变量方程
% 	      \item 齐次方程
% 	      \item 一阶线性方程
% 	    \end{itemize}
	  \vspace{1em}
	  \end{itemize}
	  \item {\bf  课后作业:}
	  \begin{itemize}
	    \item {\b 习题8.2:3,4,5(4-6),8(1-2),11,12,14,19,21}
	  \end{itemize}
	\end{enumerate}
\end{frame}

\begin{frame}[<+->]{复习与回顾}
	\linespread{1.5}
	\begin{enumerate}
	  \item {\bf 向量及其表示:}长度+方向
	  \item {\bf 向量运算:}
	  \begin{itemize}
	    \item 线性运算:平移、伸缩
	    \item 点乘:投影
	    \item 叉乘:右手法则、平行四边形面积
	    \item 混合积:体积
	  \end{itemize}
	\end{enumerate}
\end{frame}

\begin{frame}{课堂练习}
	\linespread{1.2}
	\begin{exampleblock}{判断正误\hfill}\pause
		\begin{enumerate}
		  \item $\bm{a}\cdot\bm{a}\cdot\bm{a}=\bm{a^3}$
		  \quad\pause(\;$\alert{\times}$\;)\pause
		  \item $\bm{a}\ne 0$时,$\df{\bm{a}}{\bm{a}}=1$
		  \quad\pause(\;$\alert{\times}$\;)\pause
		  \item $\bm{a}(\bm{a}\cdot\bm{b})=\bm{a}^2\bm{b}$
		  \quad\pause(\;$\alert{\times}$\;)\pause
		  \item $(\bm{a}\cdot\bm{b})^2=\bm{a}^2\bm{b}^2$
		  \quad\pause(\;$\alert{\times}$\;)\pause
		  \item $(\bm{a}+\bm{b})\times(\bm{a}-\bm{b})=\bm{a}\times\bm{a}
		  -\bm{b}\times\bm{b}=0$
		  \quad\pause(\;$\alert{\times}$\;)\pause
		  \item $\bm{a}\ne
		  0$时,$\bm{a}\cdot\bm{b}=\bm{a}\cdot\bm{c}\Rightarrow\bm{b}=\bm{c}$
		  \quad\pause(\;$\alert{\times}$\;)\pause
		  \item $\bm{a}\ne
		  0$时,$\bm{a}\times\bm{b}=\bm{a}\times\bm{c}\Rightarrow\bm{b}=\bm{c}$
		  \quad\pause(\;$\alert{\times}$\;)
		\end{enumerate}
	\end{exampleblock}
\end{frame}

\section{空间平面及其方程}

\begin{frame}{平面方程}
	\linespread{1.5}\pause 
	\ba{问题:}如何表示空间中的一个平面?\pause 
	
	\hspace{3em}\alert{(如何在空间中确定一个平面?)}\pause 
	
	\begin{enumerate}
	  \item {\bf 点法式方程}\pause 
	  \item {\bf 一般式方程}\pause 
	  \item {\bf 三点式方程}\pause 
	  \item {\bf 截距式方程}
	\end{enumerate}
% 	\begin{block}{{\bf title}\hfill}
% 		123
% 	\end{block}
\end{frame}

\begin{frame}{1、点法式方程}
	\linespread{1.5}\pause 
	\ba{给定平面上一点和一个法向量,可以唯一确定该平面}\pause 
	
	{\bf 已知:}平面上一点$M_0(\bm{r}_0)$,法向量$\bm{n}$,求平面上任一点$M(\bm{r})$满足的方程\pause 
	
	$$\alert{\bm{n}\cdot(\bm{r}-\bm{r}_0)=0}$$\pause 
	
	\vspace{-1em}
	\begin{exampleblock}{{\bf 例1}\hfill}
		求过点$M_0(2,0,-1)$,垂直于$\bm{n}=(4,2,-3)$的平面方程。
	\end{exampleblock}
\end{frame}

\begin{frame}{2、一般式方程}
	\linespread{1.5}\pause 
	$$\bm{n}\cdot(\bm{r}-\bm{r}_0)=0$$\pause 
	设$\bm{n}=(A,B,C),M_0=(x_0,y_0,z_0),M=(x,y,z)$,\pause 则有
	$$A(x-x_0)+B(y-y_0)+C(z-z_0)=0,$$
	或\pause 
	$$\alert{Ax+By+Cz+D=0}$$\pause 
	\ba{在一般式方程中,$(A,B,C)$为平面的法向量!}
\end{frame}

\begin{frame}{2、三点式方程}
	\linespread{1.5}\pause 
	\ba{不共线三点可唯一确定一个平面}\pause 
	
	{\bf 已知:}已知平面上三点$A(\bm{r}_1),B(\bm{r}_2),C(\bm{r}_3)$,
	求平面上任一点$M(\bm{r})$满足的方程\pause 
	$$\alert{[(\bm{r}_2-\bm{r}_1)\times(\bm{r}_3-\bm{r}_1)]\cdot(\bm{r}-\bm{r}_1)=0}$$\pause
	\ba{即:}$AB,AC$和$AM$三线共面
\end{frame}

\begin{frame}
	\linespread{1.5}
	$$\alert{[(\bm{r}_2-\bm{r}_1)\times(\bm{r}_3-\bm{r}_1)]\cdot(\bm{r}-\bm{r}_1)=0}$$\pause
	设$\bm{r}_i=(x_i,y_i,z_i),i=1,2,3,\bm{r}=(x,y,z)$,\pause 上式即为
	$$\alert{\left|\begin{array}{ccc}
	x-x_1 & y-y_1 & z-z_1\\
	x_2-x_1 & y_2-y_1 & z_2-z_1\\
	x_3-x_1 & y_3-y_1 & z_3-z_1\\
	\end{array}\right|
	=0}
	$$\pause 
	\begin{exampleblock}{{\bf 例2}\hfill}
		求过点三点$P(a,0,0),Q(0,b,0),R(0,0,c),abc\ne0$的平面方程。
	\end{exampleblock}
\end{frame}

\begin{frame}{3、截距式方程}
	\linespread{1.5}
	\begin{exampleblock}{{\bf 例2}\hfill}
		求过三点$P(a,0,0),Q(0,b,0),R(0,0,c),abc\ne0$的平面方程。
	\end{exampleblock}\pause 
	{\bf 已知:}某平面在三坐标轴上的截距分别为$a,b,c$,且$abc\ne 0$,\pause 则该平面方程为
	$$\alert{\df xa+\df yb+\df zc=1}$$
\end{frame}

\section{空间直线及其方程}

\begin{frame}{直线方程}
	\linespread{1.5}\pause 
	\ba{问题:}如何表示空间中的一条直线?\pause 
	\begin{enumerate}
	  \item {\bf 点向式方程}\pause 
	  \item {\bf 对称式(标准式)方程}\pause 
	  \item {\bf 一般式方程}\pause 
	  \item {\bf 两点式方程}
	\end{enumerate}
% 	\begin{block}{{\bf title}\hfill}
% 		123
% 	\end{block}
\end{frame}

\begin{frame}{1、点向式方程}
	\linespread{1.2}\pause 
	\ba{给定直线上一点及其方向,可唯一确定该直线}\pause 
	
	{\bf 已知:}直线上一点$M_0(\bm{r}_0)$,方向向量$\bm{s}$,求直线上
	任一点$M(\bm{r})$满足的方程\pause 
	$$\alert{\bm{r}-\bm{r}_0=t\bm{s}\quad(t\in\mathbb{R})}$$\pause 
	设$\bm{s}=(m,n,p),\bm{r}_0=(x_0,y_0,z_0),\bm{r}=(x,y,z)$,\pause 则
	$$
		\alert{\left\{\begin{array}{l}
			x=x_0+mt,\\
			y=y_0+nt,\quad (t\in\mathbb{R})\\
			z=z_0+pt.
		\end{array}\right.}
	$$
\end{frame}

\begin{frame}{2、对称式(标准式)方程}
	\linespread{1.2}
	$$
		\alert{\left\{\begin{array}{l}
			x=x_0+mt,\\
			y=y_0+nt,\quad (t\in\mathbb{R})\\
			z=z_0+pt.
		\end{array}\right.}
	$$\pause 
	消去参数$t$,则有
	$$
		\alert{\df{x-x_0}m=\df{y-y_0}n=\df{z-z_0}p}
	$$\pause 
	\vspace{-1em}
	\begin{exampleblock}{{\bf 例3}\hfill}
		求过原点,与三个坐标轴正向夹角相同的直线方程。
	\end{exampleblock}
\end{frame}

\begin{frame}{3、一般式方程}
	\linespread{1.2}
	$$
		\alert{\df{x-x_0}m=\df{y-y_0}n=\df{z-z_0}p}
	$$\pause 
	任意取两个部分结合,化简可得
	$$
		\alert{\left\{\begin{array}{l}
			A_1x+B_1y+C_1z=D_1\\
			A_2x+B_2y+C_2z=D_2
		\end{array}\right.}
	$$\pause 
	\ba{即:所求直线为两个平面的交线,两个平面的法向量分别为$(A_1,B_1,C_1)$
	和$(A_2,B_2,C_2)$}
\end{frame}

\begin{frame}{4、两点式方程}
	\linespread{1.2}\pause 
	\ba{空间两点可以唯一确定一条直线}\pause 
	
	{\bf 已知:}直线上两点$M_1(\bm{r_1}),M_2(\bm{r_2})$,求直线上任意一点
	$M(\bm{r})$所满足的方程\pause 
	$$\alert{\bm{r}-\bm{r}_1=t(\bm{r}_2-\bm{r}_1)\quad (t\in\mathbb{R})}$$\pause 
	设$\bm{r}_i=(x_i,y_i,z_i),i=1,2,\bm{r}=(x,y,z)$,则
	$$\alert{\df{x-x_1}{x_2-x_1}=\df{y-y_1}{y_2-y_1}=\df{z-z_1}{z_2-z_1}}$$
\end{frame}

\begin{frame}
	\linespread{1.2}\pause 
	\begin{exampleblock}{{\bf 例4}\hfill}
		化直线的一般式方程为标准式方程
		$$\left\{\begin{array}{l}
			3x+2y+z=6\\
			2x-3z=5
		\end{array}\right.$$
	\end{exampleblock}\pause 
	\begin{exampleblock}{{\bf 例5}\hfill}
		已知直线$L$过点$M(3,-1,0)$,且平行于直线
		$$\left\{\begin{array}{l}
			2x-y+3z=0\\
			y=2
		\end{array}\right.$$
		求$L$的方程。
	\end{exampleblock}
\end{frame}

\section{点、直线与平面的位置关系}

\begin{frame}{点、直线与平面的位置关系}
	\linespread{1.5}\pause 
	\ba{如何用向量或坐标的形式表示空间中各种几何对象间的位置关系?}\pause 
	\begin{enumerate}
	  \item {\bf 点到平面的距离}\pause 
	  \item {\bf 点到直线的距离}\pause 
	  \item {\bf 两平面的夹角}\pause 
	  \item {\bf 两直线的夹角}\pause 
	  \item {\bf 直线与平面的位置关系}
	\end{enumerate}
\end{frame}

\begin{frame}{1、点到平面的距离}
	\linespread{1.2}\pause 
	{\bf 问题:}求点$P(x_0,y_0,z_0)$到平面$\pi:Ax+By+Cz$
	
	$+D=0$的距离$d$\pause 
	
	\ba{投影法:}$d=$平面上任一点$M$到$P$的连线在平面法向量$\bm{n}$上的投影长度\pause 
	$$d=(\bm{MP})_{\bm{n}}=\df{|\bm{MP}\cdot\bm{n}|}{|\bm{n}|}$$\pause 
	即
	$$\alert{d=\df{|Ax_0+By_0+Cz_0+D|}{\sqrt{A^2+B^2+C^2}}}$$
\end{frame}

\begin{frame}
	\linespread{1.2}
	\begin{exampleblock}{{\bf 例6}\hfill}
		设$a,b,c$分别为某平面在三个坐标轴上的截距,$d$为其到原点的距离,证明:
		$$\df1{a^2}+\df1{b^2}+\df1{c^2}=\df1{d^2}$$
	\end{exampleblock}\pause 
	$$\alert{d=\df{|Ax_0+By_0+Cz_0+D|}{\sqrt{A^2+B^2+C^2}}}$$
\end{frame}

% \begin{frame}
% 	\linespread{1.2}
% 	\begin{exampleblock}{{\bf 例6}\hfill}
% 		已知点$P(1,3,2)$和平面$\pi:7x-4y+4z+15=0$
% 		\begin{enumerate}
% 		  \item 求$P$到$\pi$的距离;
% 		  \item 求$P$关于$\pi$的对称点$Q$的坐标。
% 		\end{enumerate}
% 	\end{exampleblock}\pause 
% 	$$\alert{d=\df{|Ax_0+By_0+Cz_0+D|}{\sqrt{A^2+B^2+C^2}}}$$
% \end{frame}

\begin{frame}{2、点到直线的距离}
	\linespread{1.2}\pause
	{\bf 问题:}求点$P(x_0,y_0,z_0)$到直线
	$$\df{x-x_1}m=\df{y-y_1}n=\df{z-z_1}p$$
	的距离$d$\pause 
	
	\ba{利用向量叉乘的几何意义:}记$\bm{s}=(m,n,p)$\pause 
	$$\alert{d=\df{|\bm{MP}\times\bm{s}|}{|\bm{s}|}}$$
\end{frame}

\begin{frame}
	\linespread{1.2}
	\begin{exampleblock}{{\bf 例7}\hfill}
		已知点$P(3,1,-4)$和直线
		$$L:\df{x+1}2=\df{y-4}{-2}=\df{z-1}{1}$$
		\vspace{-1em}
		\begin{enumerate}
		  \item 求$P$到$L$的距离;
		  \item 求$P$在$L$上的垂足$Q$的坐标;
		  \item 设$R(1,2,3)$在$L$上的垂足为$N$,求$QN$的长度。
		\end{enumerate}
	\end{exampleblock}
\end{frame}

\begin{frame}{3、两平面的夹角}
	\linespread{1.2}\pause
	{\bf 问题:}求两平面$\pi_i:A_ix+B_iy+C_iz+D_i=0,i=1,2$
	的夹角$\theta$\pause 
	
	记$\bm{n}_i=(A_i,B_i,C_i),i=1,2$,则
	$$\alert{\cos\theta=\df{\bm{n}_1\cdot\bm{n}_2}{|\bm{n}_1||\bm{n}_2|}\pause 
	=\df{A_1A_2+B_1B_2+C_1C_2}{\sqrt{A_1^2+B_1^2+C_1^2}\sqrt{A_2^2+B_2^2+C_2^2}}}$$
	\begin{itemize}
	  \item $\pi_1\perp\pi_2\pause \Leftrightarrow\bm{n}_1\perp\bm{n}_2\pause 
	  \Leftrightarrow A_1A_2+B_1B_2+C_1C_2=0$\pause 
	  \item $\pi_1//\pi_2\pause \Leftrightarrow\bm{n}_1//\bm{n}_2\pause 
	  \Leftrightarrow\df{A_1}{A_2}=\df{B_1}{B_2}=\df{C_1}{C_2}$
	\end{itemize}
\end{frame}

\begin{frame}
	\linespread{1.2}
	\begin{exampleblock}{{\bf 例8}\hfill}
		已知平面过点$M_1(1,3,-2),M_2(3,0,2)$,且与平面
		$\pi:2x+y+3z+5=0$垂直,求该平面的方程。
	\end{exampleblock}
\end{frame}

\begin{frame}{4、两直线的夹角}
	\linespread{1.2}\pause 
	{\bf 问题:}求两直线
	$$L_i:\df{x-x_i}{m_i}=\df{y-y_i}{n_i}=\df{z-z_i}{p_i},\;i=1,2$$
	的夹角$\theta$\pause 
	
	记$\bm{s}_i=(m_i,n_i,p_i),i=1,2$,则
	$$\alert{\cos\theta=\df{\bm{s}_1\cdot\bm{s}_2}{|\bm{s}_1||\bm{s}_2|}}$$
\end{frame}

\begin{frame}
	\linespread{1.2}
	\begin{exampleblock}{{\bf 例9}\hfill}
		已知直线
		$$L_1:\df{x-1}1=\df{y+1}2=\df{z-1}{\lambda},$$
		$$L_2:x+1=y-1=z,$$
		当$\lambda$取何值时,二者垂直?
	\end{exampleblock}
\end{frame}

\begin{frame}{5、直线与平面的位置关系}
	\linespread{1.2}\pause 
	已知直线和平面:
	$$L:\df{x-x_0}{m}=\df{y-y_0}{n}=\df{z-z_0}{p},$$
	$$\pi:Ax+By+Cz+D=0.$$\pause 
	记$\bm{s}=(m,n,p),\bm{n}=(A,B,C)$,\pause 则
% 	\vspace{1ex}
	\begin{enumerate}
	  \item {\bf 夹角:}\pause
	  $\sin\theta=\df{|\bm{s}\cdot\bm{n}|}{|\bm{s}||\bm{n}|}$\pause 
	  \item $L\perp\pi\pause \Leftrightarrow\bm{s}//\bm{n}\pause 
	  \Leftrightarrow \df mA=\df nB=\df pC$\pause 
	  \item $L//\pi\pause \Leftrightarrow\bm{s}\perp\bm{n}\pause 
	  \Leftrightarrow mA+nB+pC=0$
	\end{enumerate}
\end{frame}

\begin{frame}
	\linespread{1.2}
	\begin{exampleblock}{{\bf 例10}\hfill}
		已知平面$\pi$过点$M_0(2,1,3)$和直线
		$$L:\df{x+1}3=\df{y-2}2=\df{z-3}{5},$$
		求$\pi$的方程。
	\end{exampleblock}
\end{frame}

\begin{frame}
	\linespread{1.2}
	\begin{exampleblock}{{\bf 例11}}
		已知平面$\pi$和直线$l$:
		$$\pi:\;Ax+By+Cz+D=0,$$
		$$L:\;\df{x-x_0}m=\df{y-y_0}n=\df{z-z_0}p$$
		问:在什么条件下,
		\begin{enumerate}
		  \item $L$在$\pi$上;
		  \item $L$与$\pi$有且仅有一个交点。
		\end{enumerate}
	\end{exampleblock}
\end{frame}

\begin{frame}{小结}
	\linespread{1.5}
	\begin{enumerate}\pause 
	  \item {\bf 平面及其方程:}\pause 
	  \begin{itemize}
	    \item 点法式、一般式、三点式、截距式\pause 
	  \end{itemize}
	  \item {\bf 直线及其方程:}\pause 
	  \begin{itemize}
	    \item 点向式、对称式、标准式、一般式、两点式\pause 
	  \end{itemize}
	  \item {\bf 点、直线与平面的位置关系}\pause 
	  \begin{itemize}
	    \item 距离、夹角、垂直、平行\pause 
	  \end{itemize}
	\end{enumerate}
	\vspace{-1em}
	\begin{center}
		\ba{如何用向量运算表示和推导空间对象几何关系?}
	\end{center}
	
\end{frame}

\begin{frame}[<+->]{课堂讨论}
	\linespread{1.2}
	\begin{enumerate}
	  \item {\bf 如何判断两个平面相交?}
	  \begin{itemize}
	    \item \alert{若不平行,必相交}
	  \end{itemize}
	  \item {\bf 如何判断直线和平面是否相交?}
	  \begin{itemize}
	    \item \alert{若不平行,必相交}
	    \item \alert{若平面不包含直线,且二者平行,不相交}
	  \end{itemize}
	  \item {\bf 如何判断两直线是否相交?}
	  \begin{itemize}
	    \item \alert{将一条直线的参数方程代入另一直线的方程,
	    判断是否有解}
	    \item \alert{两直线距离是否为零}
	    \item \alert{过一直线做另一直线的平行面,求该面与后者的距离}
	  \end{itemize}
	  \item \ldots\ldots
	\end{enumerate}
\end{frame}

\begin{frame}{补充例题}
	\linespread{1.2}\pause 
	\begin{exampleblock}{{\bf 例12}\hfill}
		已知一向量的模为$2$,且与$x$轴和$y$轴的夹角相同,与$z$轴的夹角是
		它们的两倍,求此向量。
	\end{exampleblock}
	\bigskip\pause 
	\begin{exampleblock}{{\bf 例13}\hfill}
		设$\bm{a},\bm{b}$均为非零向量,且$|\bm{b}|=1$,二者夹角为$\pi/3$,求
		$$\limx{0}\df{|\bm{a}+x\bm{b}|-|\bm{a}|}{x}$$
	\end{exampleblock}
\end{frame}

\begin{frame}
	\linespread{1.2}
	\begin{exampleblock}{{\bf 例14}\hfill}
		已知平行四边形的两对角线向量分别为$\bm{A}=\bm{m}+2\bm{n}$,
		
		$\bm{B}=2\bm{m}-4\bm{n}$,其中$|\bm{m}|=1,|\bm{n}|=2$,
		$\bm{m}$和$\bm{n}$的夹角为$\pi/6$,求该平行四边形的面积。
	\end{exampleblock}
	\bigskip\pause 
	\begin{exampleblock}{{\bf 例15}\hfill}
		试用向量方法证明正弦定理
		$$\df{a}{\sin A}=\df{b}{\sin B}=\df{c}{\sin C}$$
	\end{exampleblock}
\end{frame}

\begin{frame}
	\linespread{1.2}
	\begin{exampleblock}{{\bf 例16:}证明\hfill}
		$$\lambda(A_1x+B_1y+C_1z+D_1)+\mu(A_2x+B_2y+C_2z+D_2)=0$$
		(其中$\lambda,\mu$不全为零)为过直线
		$$L:\left\{\begin{array}{l}
			A_1x+B_1y+C_1z+D_1=0\\
			A_2x+B_2y+C_2z+D_2=0
		\end{array}\right.$$
		的{\bb 平面束}。
	\end{exampleblock}
\end{frame}

\begin{frame}
	\linespread{1.2}
	\begin{exampleblock}{{\bf 例17}\hfill}
		求过直线
		$$\left\{\begin{array}{l}
			x+5y+z=0\\
			x-z+4=0
		\end{array}\right.$$
		且与平面$\pi:x-4y-8z+12=0$的夹角为$\pi/4$的平面方程。
	\end{exampleblock}
\end{frame}

\begin{frame}
	\linespread{1.2}
	\begin{exampleblock}{{\bf 例18}\hfill}
		已知直线
		$$L_1:\left\{\begin{array}{l}
			x+y+z+1=0\\
			2x-y+3z+4=0
		\end{array}\right.
		\quad
		L_2:\left\{\begin{array}{l}
			x=-1+2t\\
			y=-t\quad(t\in\mathbb{R})\\
			z=2-2t
		\end{array}\right.
		$$
		\begin{enumerate}
		  \item 证明两直线异面;
		  \item 求两直线间的距离;
		  \item 求二者的公垂线方程。
		\end{enumerate}
	\end{exampleblock}
\end{frame}



% \begin{frame}{课后思考}
% 	\linespread{1.5}
% 	{\bf 如何推出更“复杂”的几何关系,例如}
% 	\begin{enumerate}
% 	  \item 直线与直线的距离
% 	  \item 直线和平面的夹角
% 	  \item 平行平面的距离
% 	  \item 与平面平行的直线到平面的距离
% 	  \item \ldots\ldots
% 	\end{enumerate}
% \end{frame}

% \begin{frame}{title}
% 	\linespread{1.2}
% 	\begin{block}{{\bf title}\hfill}
% 		123
% 	\end{block}
% \end{frame}