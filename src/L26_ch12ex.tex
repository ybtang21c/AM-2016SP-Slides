% !Mode:: "TeX:UTF-8"

\begin{frame}{第二十六讲、曲线与曲面积分习题课}
	\linespread{1.5}
	\begin{enumerate}
	  \item {\bf 内容与要求}
	  \begin{itemize}
		\item 曲线与曲面积分的概念
		\item 两类曲线(曲面)积分的相互转换
		\item 曲线(面)积分的计算
		\item Green公式、Gauss公式和Stokes公式
	  \vspace{1em}
	  \end{itemize}
% 	  \item {\bf  课后作业:}
% 	  \begin{itemize}
% 	    \item {\b 习题12.4:1,2,5,6,8,14}
% 	  \end{itemize}
	\end{enumerate}
\end{frame}

\section{复习与回顾}

\begin{frame}{应用背景}
	\linespread{1.5}
	\begin{enumerate}
	  \item {\bf 质量相关的问题:}质心、转动惯量、引力
	  \begin{itemize}
	    \item 二/三重积分:平面片、空间立体
	    \item 对弧长的曲线积分:平面/空间曲线
	    \item 对面积的曲面积分:空间曲面
	  \end{itemize}
	  \item {\bf 向量场相关的问题}
	  \begin{itemize}
	    \item 流量:Green公式$\to$Stokes公式
	    \item 环量(变力做功):Green公式$\to$Gauss公式
	    \item 保守场、全微分、原(势)函数
	    \item 散度、旋度
	  \end{itemize}
	\end{enumerate}
\end{frame}

\subsection{对弧长的曲线积分}

\begin{frame}[<+->]{1、对弧长的曲线积分}
	\linespread{1.2}
	\begin{enumerate}
	  \item {\bf 背景:}曲线长度、质量、质心、转动惯量、引力
	 	$$\int_{L}f(x,y)\d s$$
	  \item {\bf 弧微分:}
		$$\d s=\sqrt{(\d x)^2+(\d y)^2}=|\bm{r}'(t)|\d t$$
	  \item {\bf 计算步骤:}曲线参数化$\to$定限$\to$计算积分
	  \item \ba{注意:}\alert{化成定积分后必须确保上限大于下限}
	\end{enumerate}
\end{frame}

\subsection{对坐标的曲线积分}

\begin{frame}[<+->]{2、对坐标的曲线积分}
	\linespread{1.2}
	\begin{enumerate}
	  \item {\bf 背景:}变力做功、流(通)量、环量
	  	$$\dint_L\bm{F}\d\bm{s}=\dint_L P\d x+Q\d y$$
	  \begin{itemize}
	    \item {\bf 流量:}$\dint_L\bm{v}\cdot\bm{n}\d s=\dint_LP\d y-Q\d x$
	    \item {\bf 环量:}$\dint_L\bm{v}\cdot\bm{T}\d s=\dint_LP\d x+Q\d y$
	  \end{itemize}
	  \item {\bf 计算步骤:}曲线参数化$\to$定限$\to$计算积分
	  \item \ba{注意:}\alert{化成定积分后上、下限与曲线的方向一致}
	\end{enumerate}
\end{frame}

\subsection{对面积的曲面积分}

\begin{frame}[<+->]{3、对面积的曲面积分}
	\linespread{1.2}
	\begin{enumerate}
	  \item {\bf 背景:}曲面面积、质量、质心、转动惯量、引力
	 	$$\iint_{\Sigma}f(x,y,z)\d S$$
	  \item {\bf 面积微元:}
		$$\d
		S=\sqrt{1+(f\,'_x)^2+(f\,'_y)^2}\d\sigma_{xy}=\df{\d\sigma_{xy}}{|\cos\gamma|}$$
		$$\d S=\df{\d\sigma_{yz}}{|\cos\alpha|}=
		\df{\d\sigma_{zx}}{|\cos\beta|}
		=\df{\d\sigma_{xy}}{|\cos\gamma|}$$
	  \item {\bf 计算步骤:}投影$\to$写出面积微元$\to$计算积分
% 	  \item \ba{注意:}\alert{化成定积分后必须确保上限大于下限}
	\end{enumerate}
\end{frame}

\subsection{对坐标的曲面积分}

\begin{frame}[<+->]{4、对坐标的曲面积分}
	\linespread{1.2}
	\begin{enumerate}
	  \item {\bf 背景:}流(通)量
	 	$$\iint_{\Sigma}\bm{v}\cdot\bm{n}\d S=
	 	\iint_{\Sigma}P\d y\d z+Q\d z\d x+R\d x\d y$$
	 	$$\bm{n}\d S=(\d\sigma_{yz},\d\sigma_{zx},\d\sigma_{xy})$$
	  \item {\bf 计算步骤:}确定投影方向$\to$确定积分符号$\to$计算积分
	  \item \ba{灵活掌握两类曲面积分的相互转换}
	\end{enumerate}
\end{frame}

\subsection{Green公式$\to$Gauss公式}

\begin{frame}[<+->]{5、Green公式$\to$Gauss公式}
	\linespread{1.2}
	\begin{enumerate}
	  \item {\bf 背景:}流(通)量与散度
	  $$\oint_{\p D}\bm{v}\cdot\bm{n}\d s=
	  \int_D{\mathrm{div}\,\bm{v}}{\d\sigma}$$
	  \item \ba{注意:}\alert{$\bm{v}$的各分量在$D$内必须偏导连续}
	  \item {\bf 无源场:}$\mathrm{div}\,\bm{v}=0$
	\end{enumerate}
\end{frame}

\subsection{Green公式$\to$Stokes公式}

\begin{frame}[<+->]{6、Green公式$\to$Stokes公式}
	\linespread{1.2}
	\begin{enumerate}
	  \item {\bf 背景:}环量与旋度
	  $$\oint_{\p\Sigma}\bm{v}\cdot\bm{T}\d s=
	  \int_{\Sigma}{\mathrm{rot}\,\bm{v}}\cdot\bm{n}{\d S}$$
	  \item \alert{注意掌握旋度的行列式形式}
	  \item {\bf 无旋场:}保守场
	  $$\mathrm{rot}\,\bm{v}=0$$
	  \item {\bf 原(势)函数:}折线法、凑微分法
	\end{enumerate}
\end{frame}

\section{对称性在积分计算中的应用}

\begin{frame}{对称性在积分计算中的应用}
	\linespread{1.5}
	\begin{enumerate}
	  \item {\bf 关键词}
	  \begin{itemize}
		\item (函数的)奇偶性
		\item (区域的)对称性
		\item (符号的)对称性
		\item 定积分、重积分、曲线积分、曲面积分
	  \vspace{1em}
	  \end{itemize}
% 	  \item {\bf  课后作业:}
% 	  \begin{itemize}
% 	    \item {\b 习题12.4:1,2,5,6,8,14}
% 	  \end{itemize}
	\end{enumerate}
\end{frame}

\subsection{二重积分中的对称性}

\begin{frame}{1、二重积分中的对称性}
	\linespread{1.2}\pause
	\begin{block}{{\bf 命题1.1}(坐标对称)\hfill}
		区域$D$关于$x=0$对称,$D_1$为其中$x\geq 0$的部分,则
		$$\iint_Df(x,y)\d\sigma=\left\{\begin{array}{ll}
			0,& f\mbox{关于}x\mbox{是奇函数}\\
			2\ds\iint_{D_1}f(x,y)\d\sigma,& f\mbox{关于}x\mbox{是偶函数}
		\end{array}\right.$$
	\end{block}
	\pause
	\alert{{\bf 注:}将以上关于$x$的性质改成关于$y$的,结论同样成立}
\end{frame}

\begin{frame}
	\linespread{1.2}
	\begin{block}{{\bf 命题1.2}\hfill}
		区域$D$关于$x=0,y=0$都对称,$D_1$为其在第一象限内的部分,
		若$f(x,y)$关于$x,y$均为偶函数,则
		$$\iint_Df(x,y)\d\sigma=4\iint_{D_1}f(x,y)\d\sigma$$
	\end{block}
	\pause
	\begin{exampleblock}{{\bf 例1}\hfill}
		$$\iint_{x^2+y^2\leq 1}\df{(x+2y)^2}{1+x^2+y^2}\d\sigma$$
	\end{exampleblock}
\end{frame}

\begin{frame}
	\linespread{1.2}
	\begin{block}{{\bf 命题1.3}(轴对称)\hfill}
		区域$D$关于某条直线对称,对$D$内关于该轴对称的任意两点$M,N$,总有
		$f(M)=-f(N)$,则
		$$\iint_Df(x,y)\d\sigma=0$$
	\end{block}
	\pause
	\begin{exampleblock}{{\bf 例2}\hfill}
	$D:0\leq x\leq 1,0\leq y\leq 1$,求
		$$\iint_D(x+y)\mathrm{sgn}(x-y)\d\sigma$$
	\end{exampleblock}
\end{frame}

\begin{frame}
	\linespread{1.2}
	\begin{block}{{\bf 命题1.4}\hfill}
		区域$D$关于$y=x$对称,则
		$$\iint_Df(x,y)\d\sigma=\iint_Df(y,x)\d\sigma$$
	\end{block}
	\pause
	\begin{exampleblock}{{\bf 例3}\hfill}
		区域$D$为第一象限内的四分之一单位圆,求
		$$\iint_D\df{2x^2+x-y+1}{\sqrt{1-x^2-y^2}}\d\sigma$$
	\end{exampleblock}
\end{frame}

\begin{frame}
	\linespread{1.2}
	\begin{block}{{\bf 命题1.5}(轮换对称)\hfill}
		区域$D$内$x,y$地位相同,则
		$$\iint_Df(x,y)\d\sigma=\iint_Df(y,x)\d\sigma$$
	\end{block}
	\pause
	\begin{exampleblock}{{\bf 例4}\hfill}
		区域$D$为第一象限内的四分之一单位圆,$f(x)$恒为正,求
		$$\iint_D\df{a\sqrt{f(x)}+b\sqrt{f(y)}}{\sqrt{f(x)}+\sqrt{f(y)}}\d\sigma$$
	\end{exampleblock}
\end{frame}

\subsection{三重积分中的对称性}

\begin{frame}{2、三重积分中的对称性}
	\linespread{1.2}\pause
	\begin{block}{{\bf 命题2.1}(坐标对称)\hfill}
		空间区域$\Omega$关于$z=0$对称,$\Omega_1$为其中$z\geq 0$的部分,则
		$$\iiint_{\Omega}f\d V=\left\{\begin{array}{ll}
			0,& f\mbox{关于}z\mbox{是奇函数}\\
			2\ds\iiint_{\Omega_1}f\d V,& f\mbox{关于}z\mbox{是偶函数}
		\end{array}\right.$$
	\end{block}
\end{frame}

\begin{frame}
	\linespread{1.2}
	\begin{block}{{\bf 命题2.2}(平面对称)\hfill}
		区域$\Omega$关于某平面对称,对$\Omega$内关于该平面对称的任意两点$M,N$,
		总有$f(M)=-f(N)$,则
		$$\iiint_{\Omega}f(x,y,z)\d V=0$$
	\end{block}
	\pause
	\begin{block}{{\bf 命题2.3}\hfill}
		区域$\Omega$关于平面$y=x$对称,则
		$$\iiint_{\Omega}f(x,y,z)\d V=\iiint_{\Omega}f(y,x,z)\d V$$
	\end{block}
\end{frame}

\begin{frame}
	\linespread{1.2}
	\begin{block}{{\bf 命题2.4}(轮换对称)\hfill}
		若区域$\Omega$中$x,y,z$地位相同(例:$\Omega$为一球体),则积分
		$$\iiint_{\Omega}f(x,y,z)\d V$$
		中$x,y,z$的次序可以任意交换。
	\end{block}
	\pause
	\begin{exampleblock}{{\bf 例5}\hfill}
		$$\iiint_{x^2+y^2+z^2\leq R^2}\df{x^2-y^2}{(1+x^2+y^2+z^2)
		^{3/2}}\d V$$
	\end{exampleblock}
\end{frame}

\subsection{对弧长的曲线积分中的对称性}

\begin{frame}{3、对弧长的曲线积分中的对称性}
	\linespread{1.2}\pause
	\begin{block}{{\bf 命题3.1}(坐标对称)\hfill}
		曲线$L$关于$x=0$对称,$L_1$为其中$x\geq 0$的部分,则
		$$\int_{L}f(x,y)\d s=\left\{\begin{array}{ll}
			0,& f\mbox{关于}x\mbox{是奇函数}\\
			2\ds\int_{L_1}f\d s,& f\mbox{关于}x\mbox{是偶函数}
		\end{array}\right.$$
	\end{block}
	\pause
	\begin{exampleblock}{{\bf 例6}\hfill}
		$$\oint_{|x|+|y|=1}\df{x}{|x|+|y|}\d s$$
	\end{exampleblock}
\end{frame}

\begin{frame}
	\linespread{1.2}
	\begin{block}{{\bf 命题3.2}(平面对称)\hfill}
		空间曲线$L$关于$x=0$对称,$L_1$为其中$x\geq 0$的部分,则
		$$\int_{L}f(x,y,z)\d s=\left\{\begin{array}{ll}
			0,& f\mbox{关于}x\mbox{是奇函数}\\
			2\ds\int_{L_1}f\d s,& f\mbox{关于}x\mbox{是偶函数}
		\end{array}\right.$$
	\end{block}
\end{frame}

\subsection{对面积的曲面积分中的对称性}

\begin{frame}{4、对面积的曲面积分中的对称性}
	\linespread{1.2}\pause
	\begin{block}{{\bf 命题4.1}(平面对称)\hfill}
		曲线$\Sigma$关于$x=0$对称,$\Sigma_1$为其中$x\geq 0$的部分,则
		$$\iint_{\Sigma}f(x,y,z)\d S=\left\{\begin{array}{ll}
			0,& f\mbox{关于}x\mbox{是奇函数}\\
			2\ds\iint_{\Sigma_1}f\d S,& f\mbox{关于}x\mbox{是偶函数}
		\end{array}\right.$$
	\end{block}
	\pause
	\begin{exampleblock}{{\bf 例7}\hfill}
		$\Sigma$为$x^2+y^2=R^2$在$z=0$和$z=1$间的部分,求
		$$\iint_{\Sigma}\df y{x^2+y^2+z^2}\d S$$
	\end{exampleblock}
\end{frame}

\begin{frame}
	\linespread{1.2}
	\begin{block}{{\bf 命题4.2}(轮换对称)\hfill}
		曲面$\Sigma$上$x,y,z$地位相同(例:$\Sigma$为一球面),则
		$$\iint_{\Sigma}f(x,y,z)\d S$$
		中$x,y,z$的次序可以任意交换。
	\end{block}
	\pause
	\begin{exampleblock}{{\bf 例8}\hfill}
		$$\oint_{x^2+y^2+z^2=R^2}(x^2+3y^2)\d S$$
	\end{exampleblock}
\end{frame}

\subsection{对坐标的曲线、曲面积分中的对称性}

\begin{frame}{5、对坐标的曲线、曲面积分中的对称性}
	\linespread{1.2}\pause
	\begin{block}{{\bf 命题5.1}\hfill}
		空间曲面$\Sigma$关于$z=0$对称,$\Sigma_1$为其中$z\geq 0$的部分,则
		$$\iint_{\Sigma}f(x,y,z)\d x\d y=\left\{\begin{array}{ll}
			0,& f\mbox{关于}z\mbox{是偶函数}\\
			2\ds\iint_{\Sigma_1}f\d x\d y,& f\mbox{关于}z\mbox{是奇函数}
		\end{array}\right.$$
	\end{block}
	\pause
	\begin{exampleblock}{{\bf 例9}\hfill}
		$$\oiint_{x^2+y^2+z^2=R^2}\df{z^2}{x^2+3y^2+1}\d x\d y$$
	\end{exampleblock}
\end{frame}

\begin{frame}
	\linespread{1.2}\pause
	\begin{block}{{\bf 命题5.2}\hfill}
		空间曲面$\Sigma$关于$x=0$对称,$\Sigma_1$为其中$x\geq 0$的部分,则
		$$\iint_{\Sigma}f(x,y,z)\d x\d y=\left\{\begin{array}{ll}
			0,& f\mbox{关于}x\mbox{是奇函数}\\
			2\ds\iint_{\Sigma_1}f\d x\d y,& f\mbox{关于}x\mbox{是偶函数}
		\end{array}\right.$$
	\end{block}
	\pause
	\begin{exampleblock}{{\bf 例10}\hfill}
		$$\oiint_{x^2+y^2+z^2=R^2}\df{x}{x^2+3y^2+1}\d x\d y$$
	\end{exampleblock}
\end{frame}

\begin{frame}{小结}
	\linespread{1.5}
	\begin{enumerate}
	  \item {\bf 轴对称性:}坐标对称、平面对称
	  \item {\bf 轮换对称性:}$x,y,z$地位对等
	\end{enumerate}
	\pause\bigskip\hrule
	\begin{center}
		\ba{对称性:积分区域的对称性}
		
		\bigskip\pause
		\ba{根据不同类型的积分灵活应用}
	\end{center}
\end{frame}

\section{补充例题}

\subsection{填空}

\begin{frame}{一、填空}
	\linespread{2}
% 	\begin{exampleblock}{{\bf 例1}\hfill}
		\begin{enumerate}
		  \item 设$L$为曲线$x=\df{3at}{1+t^3},y=\df{3at^2}{1+t^3}$上$t$由
		  $0$到$+\infty$的一段,$a>0$,则$\dint_Lx\d y-y\d x=$\pause
		  \underline{\quad\alert{$3a^2$}\quad}\pause
		  \item $f(x)$连续可导,$L$为$(3,2/3)$到$(1,2)$的直线,
		  则$\dint_L\df{1+y^2f(xy)}y\d x+\df
		  x{y^2}[y^2f(xy)-1]\d y=$\pause
		  \underline{\quad\alert{$-4$}\quad}\pause
		  \item $\ds\iint_{z=\sqrt{a^2-x^2-y^2}}(x+y+z)\d S=$\pause
		  \underline{\quad\alert{$\pi a^3$}\quad}
		\end{enumerate}
% 	\end{exampleblock}
\end{frame}

\begin{frame}
	\linespread{2}
% 	\begin{exampleblock}{{\bf 例1}\hfill}
		\begin{enumerate}
		  \addtocounter{enumi}{3}
		  \item 设$\Sigma$为平面$x+y+z=1$在第一卦限的上侧,$f(x,y,z)$连续,则
		  $\ds\iint_{\Sigma}[f(x,y,z)+x]\d y\d z-[2f(x,y,z)-y]\d z\d x+
		  [f(x,y,z)+z]\d x\d y=$\pause
		  \underline{\quad\alert{$\df 12$}\quad}\pause
		  \item 设$\Sigma$为锥面$z=\sqrt{x^2+y^2}(0\leq z\leq 1)$的下侧,则
		  $\ds\iint_{\Sigma}x\d y\d z+2y\d z\d x+3(z-1)\d x\d y=$\pause
		  \underline{\quad\alert{$2\pi$}\quad}
		\end{enumerate}
% 	\end{exampleblock}
\end{frame}

\begin{frame}
	\linespread{2}
% 	\begin{exampleblock}{{\bf 例1}\hfill}
		\begin{enumerate}
		  \addtocounter{enumi}{5}
		  \item 设$L$是摆线$x=t-\sin t-\pi,y=1-\cos t$从$t=0$
		  到$t=2\pi$的一段,则$\dint_L\df{(x-y)\d x+(x+y)\d y}{x^2+y^2}=$\pause
		  \underline{\quad\alert{$\pi$}\quad}
		\end{enumerate}
% 	\end{exampleblock}
\end{frame}

\subsection{选择}

\begin{frame}{二、选择}
	\linespread{2}
% 	\begin{exampleblock}{{\bf 例1}\hfill}
		\begin{enumerate}
		  \addtocounter{enumi}{6}
		  \item
		  设$L_1:\df{x^2}{4}+\df{y^2}{9}=1,L_2:\df{x^2}{9}+\df{y^2}{4}=1$,
		  二者所围封闭区域分别为$D_1,D_2$,则下列正确的是\;
		  ({\;\uncover<2->{\alert{C}}\;})
%  		  \vspace{-1cm}
		  \begin{enumerate}[(A)]
		    \item $\dint_{L_1}(x+y^2)\d s=2\dint_{L_2}y^2\d s$
		    \item $\dint_{L_1}(x^2+y)\d s=2\dint_{L_2}(x^2+y)\d s$
		    \item $\ds\iint_{D_1}(x+y^3)\d\sigma=2\ds\iint_{D_2}(x+y^3)\d\sigma$
		    \item $\ds\iint_{D_1}(x^2+y)\d\sigma=2\ds\iint_{D_2}(x^2+y)\d\sigma$
		  \end{enumerate}
		\end{enumerate}
% 	\end{exampleblock}
\end{frame}

\begin{frame}
	\linespread{2}
% 	\begin{exampleblock}{{\bf 例1}\hfill}
		\begin{enumerate}
		  \addtocounter{enumi}{7}
		  \item
		  $f(x,y)$偏导连续,曲线$L:f(x,y)=1$过第二象限的点$M$
		  和第四象限的点$N$,$\Gamma$为$L$上从$M$到$N$的一段弧,则下列
		  小于零的是\;
		  ({\;\uncover<2->{\alert{B}}\;})
		  \begin{enumerate}[(A)]
		    \item $\dint_{\Gamma}f(x,y)\d x$
		    \item $\dint_{\Gamma}f(x,y)\d y$
		    \item $\ds\int_{\Gamma}f(x,y)\d s$
		    \item $\ds\int_{\Gamma}f\,'_x(x,y)\d x+f\,'_y(x,y)\d y$
		  \end{enumerate}
		\end{enumerate}
% 	\end{exampleblock}
\end{frame}

\begin{frame}
	\linespread{2}
% 	\begin{exampleblock}{{\bf 例1}\hfill}
		\begin{enumerate}
		  \addtocounter{enumi}{8}
		  \item
		  设曲面$S_1:x^2+y^2+z^2=1(z\geq
		  0)$,$S_2$为$S_1$在第一卦限中的部分,
		  则以下正确的是\;
		  ({\;\uncover<2->{\alert{C}}\;}) 
		  \begin{enumerate}[(A)]
		    \item $\ds\iint_{S_1}x\d S=4\iint_{S_2}x\d S$
		    \item $\ds\iint_{S_1}y\d S=4\iint_{S_2}x\d S$
		    \item $\ds\iint_{S_1}z\d S=4\iint_{S_2}x\d S$
		    \item $\ds\iint_{S_1}xyz\d S=4\iint_{S_2}xyz\d S$
		  \end{enumerate}
		\end{enumerate}
% 	\end{exampleblock}
\end{frame}

\begin{frame}
	\linespread{2}
% 	\begin{exampleblock}{{\bf 例1}\hfill}
		\begin{enumerate}
		  \addtocounter{enumi}{9}
		  \item 设$f(r)$二阶连续可微,$r=\sqrt{x^2+y^2+z^2}$,
		  若$\mathrm{div}(\bigtriangledown\,f(r))=0$,则$f(r)=$\;
		  ({\;\uncover<2->{\alert{B}}\;}) 
		  \begin{enumerate}[(A)]
		    \item $C_1r+C_2$
		    \item $C_1/r+C_2$
		    \item $C_1r^2+C_2$
		    \item $C_1/r^2+C_2$
		  \end{enumerate}
		  以上$C_1,C_2$为任意常数
		\end{enumerate}
% 	\end{exampleblock}
\end{frame}

\subsection{计算与解答}

\begin{frame}{三、计算与解答}
	\linespread{1.2}
	\begin{exampleblock}{{\bf 例1}\hfill}
		设$f(x)$当$x>0$时可导,$f(1)=2$,对右半平面内的任意封闭曲线$C$,
		有$\ds\oint_C4x^3y\d x+xf(x)\d y=0$
		\begin{enumerate}
		  \item 求$f(x)$;
		  \item 设$L$为从$(1,0)$到$(2,3)$的一段弧,计算
		  $$\dint_L4x^3y\d x+xf(x)\d y$$
		\end{enumerate}
	\end{exampleblock}
\end{frame}

\begin{frame}
	\linespread{1.2}
	\begin{exampleblock}{{\bf 例2}\hfill}
		已知曲线$L:\left\{\begin{array}{l}
			x^2+y^2+z^2=R^2\\ x+y+z=0
		\end{array}\right.$,
		计算曲线积分
		$$\oint_Lz^2\d s$$
	\end{exampleblock}
\end{frame}

\begin{frame}
	\linespread{1.2}
	\begin{exampleblock}{{\bf 例3}\hfill}
		函数$u(x,y),v(x,y)$在单位圆内存在一阶连续偏导数,
		$$\bm{f}(x,y)=(v(x,y),u(x,y)),$$
		$$\bm{g}(x,y)=\left(u'_x-u'_y,v'_x-v'_y\right),$$
		在单位圆上,$u(x,y)=x,v(x,y)=1$,求
		$$\iint_{x^2+y^2\leq 1}\bm{f}\cdot\bm{g}\d\sigma$$
	\end{exampleblock}
\end{frame}

\begin{frame}
	\linespread{1.2}
	\begin{exampleblock}{{\bf 例4}\hfill}
		设$\Sigma$为曲面$z=\sqrt{x^2+y^2}$及平面$z=1$和$z=2$
		所围立体的外表面,求
		$$\oiint_{\Sigma}\sqrt{x^2+y^2}e^z(\d y\d z+\d z\d x+\d x\d y)$$
	\end{exampleblock}
\end{frame}

\begin{frame}
	\linespread{1.2}
	\begin{exampleblock}{{\bf 例5}\hfill}
		设$\Sigma$为$x^2+y^2=R^2$及平面$z=\pm R\,(R>0)$所围立体的外表面,求
		$$\oiint_{\Sigma}\df{x\d y\d z+y^2\d z\d x+z^2\d x\d y}{x^2+y^2+z^2}$$
	\end{exampleblock}
\end{frame}

\begin{frame}
	\linespread{1.2}
	\begin{exampleblock}{{\bf 例6}\hfill}
		设$\Sigma$为$2x^2+2y^2+z^2=4$的外侧,求
		$$\oiint_{\Sigma}\df{x\d y\d z+y\d z\d x+z\d x\d y}{(x^2+y^2+z^2)^{3/2}}$$
	\end{exampleblock}
\end{frame}

\begin{frame}
	\linespread{2}
	\begin{exampleblock}{{\bf 例7}\hfill}
		在变力$\bm{F}=(yz,zx,xy)$的作用下,质点由原点沿直线运动到椭球面
		$\df{x^2}{a^2}+\df{y^2}{b^2}+\df{z^2}{c^2}=1$上第一卦限
		中的某点$M$,问$M$在何位置时,$\bm{F}$所做的功最大,并求出功的最大值。
	\end{exampleblock}
\end{frame}

%=====================================

% \begin{frame}{title}
% 	\linespread{1.2}
% 	\begin{exampleblock}{{\bf title}\hfill}
% 		123
% 	\end{exampleblock}
% \end{frame}
% 
% \begin{frame}{title}
% 	\linespread{1.2}
% 	\begin{block}{{\bf title}\hfill}
% 		123
% 	\end{block}
% \end{frame}