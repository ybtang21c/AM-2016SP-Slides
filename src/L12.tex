% !Mode:: "TeX:UTF-8"

\begin{frame}{课堂练习}
	\linespread{1.2}
	\begin{exampleblock}{{\bf 例1}\hfill}
		设$z=f(x,x^2+y^2,x^2-y^2)$,其中$f$具有二阶连续偏导函数,求$\df{\p^2 z}{\p x^2}$和
		$\df{\p^2 z}{\p x\p y}$
	\end{exampleblock}
\end{frame}

\begin{frame}{第十二讲、隐函数的偏导数}
	\linespread{1.5}
	\begin{enumerate}
	  \item {\bf 内容与要求}{\color{blue}( \S10.3.2-\S10.3.4 )}
	  \begin{itemize}
	    \item 理解多元函数一阶微分的形式不变性
	    \item 掌握多元隐函数的求导法则
	    \item 熟练掌握隐函数求导的几何应用
	  \vspace{1em}
	  \end{itemize}
	  \item {\bf  课后作业:}
	  \begin{itemize}
	    \item {\b 习题10.3:7,9,12,14,16}
	  \end{itemize}
	\end{enumerate}
\end{frame}

\begin{frame}[<+->]{复习与回顾}
	\linespread{1.5}
	\begin{enumerate}
	  \item {\bf 多元函数的可微与局部线性化}
	  \begin{itemize}
	    \item 可微与连续的关系
	    \item 可微与偏导数的关系
	    \item 微分形式
	  \end{itemize}
	  \item {\bf 复合函数偏导数的计算}
	  \begin{itemize}
	    \item 链式法则:并列+分支
	    \item 利用微分运算求偏导数
	  \end{itemize}
	\end{enumerate}
\end{frame}

\section{一阶微分的形式不变性}

\begin{frame}{一阶微分的形式不变性}
	\linespread{1.2}\pause 
	\ba{一元函数的微分形式不变性:}对于一元函数$f(x)$,无论$x$是自变量
	还是中间变量,其一阶微分形式都是
	$$\d y=f'(x)\d x$$
	\pause %\vspace{-1em}
	\ba{多元函数的微分形式不变性:}$z=f(\varphi(x,y),\psi(x,y))$的全微分
	$$\alert{\d z=\df{\p z}{\p x}\d x+\df{\p z}{\p y}\d y
	=\df{\p z}{\p u}\d u+\df{\p z}{\p v}\d v}$$
	{\bb 无论$u,v$是自变量还是中间变量,其微分形式都一样}
\end{frame}

\begin{frame}[<+->]{多元函数全微分的运算法则}
	\linespread{1.5}
	\begin{enumerate}
	  \item $\d(u\pm v)=\d u\pm \d v$
	  \item $\d(cu)=c\d u\;(C\in\mathbb{R})$
	  \item $\d(uv)=v\d u+u\d v$
	  \item $\d\left(\df uv\right)=\df{u\d v-v\d u}{v^2}$
	  \item $\d\left(\df 1v\right)=-\df{\d v}{v^2}$
	\end{enumerate}
\end{frame}

\begin{frame}{利用微分运算求偏导数}
	\linespread{1.2}
	\begin{exampleblock}{{\bf 例2}\hfill}
		求$z=\arctan\df yx$的全微分。
	\end{exampleblock}\pause 
	\begin{exampleblock}{{\bf 例2'}\hfill}
		设$z=\arctan\df yx$,求$\df{\p z}{\p x}$和$\df{\p z}{\p y}$。
	\end{exampleblock}
\end{frame}

\begin{frame}
	\linespread{1.5}
	\begin{exampleblock}{{\bf 例3}\hfill}
		求解微分方程:\pause 
		\begin{enumerate}
		  \item $(e^{x+y}-e^x)\d x+(e^{x+y}+e^y)\d y=0$\pause 
		  \item $x\d y-y\d x=0$
		\end{enumerate}
	\end{exampleblock}
\end{frame}

\section{隐函数的偏导数}

\begin{frame}{隐函数的偏导数}
	\linespread{1.2}
	{\b {\bf 隐函数:}蕴含在代数方程中的函数关系}\pause 
	
	例如:
	\begin{itemize}
	  \item $x^2+y^2+z^2=4$\pause 
	  \item $2^{xy}=x+y$\pause 
	  \item $\left\{\begin{array}{l}
	  	x^2+y^2+z^2=4\\
	  	(x-1)^2+y^2=1
	  \end{array}\right.$\pause 
	  \item \ldots
	\end{itemize}
	
	\ba{问题:如何对这种隐含的函数关系求导?}
\end{frame}

\begin{frame}{隐函数中的函数关系}
	\linespread{1.2}\pause 
	\begin{enumerate}
	  \item {\bf 一个方程所确定的隐函数}\pause 
	  $$F(x,y)=0\pause \quad \Rightarrow \quad y=f(x)\pause $$
	  $$F(x,y,z)=0\pause \quad \Rightarrow \quad z=f(x,y)\pause $$
	  \item {\bf 方程组所确定的隐函数}\pause 
	  $$\left.\begin{array}{l}
	  	F(x,y,z)=0\\ G(x,y,z)=0
	  \end{array}\right\}\pause \quad \Rightarrow \quad 
	  \left\{\begin{array}{l}
	  	y=f(x)\\ z=g(x)
	  \end{array}\right. $$
	\end{enumerate}
\end{frame}

\begin{frame}{1、一个方程所确定的隐函数}
	\linespread{1.2}\pause 
	\begin{block}{{\bf 定理10.3.2}(隐函数存在定理)\hfill}
		若函数$F(x,y)$满足:\pause 
		\begin{enumerate}
		  \item $F(x_0,y_0)=0$\pause 
		  \item $F(x,y)$在$(x_0,y_0)$附近偏导数连续\pause 
		  \item $F_y'(x_0,y_0)\ne 0$\pause 
		\end{enumerate}
		则:在$x_0$附近有连续可导的函数$y=f(x)$,\pause 满足$F(x,f(x))=0$,\pause 且
		  $$\alert{f'(x)=-\df{F'_x}{F'_y}}$$
	\end{block}
\end{frame}

\begin{frame}
	\linespread{1.5}
	\begin{exampleblock}{{\bf 例3}\hfill}
		\begin{enumerate}
		  \item 设$2^{xy}=x+y$,求$\df{\d y}{\d x}$\pause 
		  \item 设$x^2+y^2+z^2=3xyz$,求$\df{\p z}{\p x},\df{\p z}{\p y}$\pause 
		  \item 设$x^2y-e^z=z$,求$\df{\p z}{\p x},\df{\p z}{\p y}$
		  和$\df{\p^2z}{\p x\p y}$
		\end{enumerate}
	\end{exampleblock}
\end{frame}

\begin{frame}{2、方程组所确定的隐函数}
	\linespread{1.2}\pause 
	\begin{exampleblock}{{\bf 例4}\hfill}
		设$y=y(x),z=z(x)$由以下方程组确定,求$\df{\d y}{\d x},\df{\d z}{\d
		x}$
		$$\left\{\begin{array}{l} x+y+z=0\\ x^2+y^2+z^2=1
		\end{array}\right.$$
	\end{exampleblock}\pause 
	{\small\vspace{-1em}
	$$
		\alert{\df{\d y}{\d x}=\df{\left|\begin{array}{cc}
			-1 & 1\\ -x & z
		\end{array}\right|}{\left|\begin{array}{cc}
			1 & 1\\ y & z
		\end{array}\right|}=\df{x-z}{z-y},}
		\quad\pause 
		\alert{\df{\d z}{\d x}=\df{\left|\begin{array}{cc}
			1 & -1\\ y & -x
		\end{array}\right|}{\left|\begin{array}{cc}
			1 & 1\\ y & z
		\end{array}\right|}=\df{y-x}{z-y}}
	$$
	}
\end{frame}

\begin{frame}
	\linespread{1.2}
	\begin{exampleblock}{{\bf 例5}\hfill}
		设$u=u(x,y),v=v(x,y)$由以下方程组确定,求$u'_x,u'_y$和$v'_x,v'_y$
		$$
			\left\{\begin{array}{l}
				F(x,y,u,v)=0\\ G(x,y,u,v)=0
			\end{array}\right.
		$$
	\end{exampleblock}\pause 
	{\vspace{-1em}
	$$
		\alert{\df{\p u}{\p x}=\df{\left|\begin{array}{cc}
			F'_x & F'_v\\ G'_x & G'_v
		\end{array}\right|}{\left|\begin{array}{cc}
			F'_u & F'_v\\ G'_u & G'_v
		\end{array}\right|},}
		\quad
		\alert{\df{\p v}{\p x}=\df{\left|\begin{array}{cc}
			F'_u & F'_x\\ G'_u & G'_x
		\end{array}\right|}{\left|\begin{array}{cc}
			F'_u & F'_v\\ G'_u & G'_v
		\end{array}\right|}}
	$$
	}
\end{frame}

\begin{frame}
	\linespread{1.2}
	\begin{exampleblock}{{\bf 例5}\hfill}
		设$u=u(x,y),v=v(x,y)$由以下方程组确定,求$u'_x,u'_y$和$v'_x,v'_y$
		$$
			\left\{\begin{array}{l}
				F(x,y,u,v)=0\\ G(x,y,u,v)=0
			\end{array}\right.
		$$
	\end{exampleblock}
	\ba{隐函数存在条件}
	\begin{enumerate}
	  \item $F,G$的偏导数连续
	  \item {\bb Jacobi行列式:}
	  $$\alert{J=\df{\p(F,G)}{\p(u,v)}=\left|\begin{array}{cc}
			F'_u & F'_v\\ G'_u & G'_v
		\end{array}\right|\ne 0}$$
	\end{enumerate}
\end{frame}

\section{隐函数求导在几何上的应用}

\begin{frame}{1、空间曲面的切平面和法线}
	\linespread{1.2}\pause 
	{\bf 已知:}光滑曲面$S:F(x,y,z)=0$,$P(x_0,y_0,z_0)\in S$\pause 
	\begin{itemize}
	  \item {\bb 过$P$的法向量:}\pause $\alert{(F'_x(P),F'_y(P),F'_z(P))}$\pause 
	  \item {\bb 切平面:}
	  	$$F'_x(P)(x-x_0)+F'_y(P)(y-y_0)+F'_z(P)(z-z_0)=0$$\pause 
	  \item {\bb 法线:}
	  	$$\df{x-x_0}{F'_x(P)}=\df{y-y_0}{F'_y(P)}=\df{z-z_0}{F'_z(P)}$$
	\end{itemize}
\end{frame}

\begin{frame}
	\linespread{1.2}
	\begin{block}{{\bf 推论}\hfill}
		若曲面方程为$z=f(x,y)$,\pause 记
		$$F(x,y,z)=f(x,y)-z,$$\pause 则
		$$F'_x=f'_x,\;F'_y=f'_y,\;F_z=-1,$$
		\pause 若$f(x,y)$在$(x_0,y_0)$偏导数连续,则曲面在该点存在唯一的切平面和法线。
	\end{block}
\end{frame}

\begin{frame}{2、空间曲线的切线和法平面}
	\linespread{1.2}\pause 
	{\bf 已知:}曲线的一般式方程
	$$\Gamma: \left\{\begin{array}{l}
		F(x,y,z)=0\\ G(x,y,z)=0
	\end{array}\right.,$$
	点$M\in\Gamma$\pause 
	\begin{itemize}
	  \item {\bb $P$处的切向量:}\pause 
	  $$\alert{\left(1,\left.\df 1J\df{\p(F,G)}{\p(z,x)}\right|_M,
	  \left.\df 1J\df{\p(F,G)}{\p(x,y)}\right|_M\right)}$$
	\end{itemize}
\end{frame}

\begin{frame}
	\linespread{1.5}
	\begin{itemize}
	  \item {\bb 法平面方程:}\pause 
	  $$\alert{
	  	\left|
	  		\begin{array}{ccc}
	  			x-x_0 & y-y_0 & z-z_0\\
	  			F'_x(M) & F'_y(M) & F'_z(M)\\
	  			G'_x(M) & G'_y(M) & G'_z(M)
	  		\end{array}
	  	\right|=0
	  }$$
	\end{itemize}\pause 
	\begin{exampleblock}{{\bf 例6}\hfill}
		求曲线$x^2+y^2+z^2=6,x+y+z=0$在点$M(1,-2,1)$处的切线和法平面方程。
	\end{exampleblock}
\end{frame}

\begin{frame}[<+->]{小结}
	\linespread{1.5}
	\begin{enumerate}
	  \item {\bf 一阶微分的形式不变性}
	  \item {\bf 隐函数求导法则}
	  \begin{itemize}
	    \item 必要条件:Jacobi行列式:$J\ne 0$
	  \end{itemize}
	  \item {\bf 隐函数求导的几何应用}
% 	  \begin{itemize}
% 	    \item 与向量值函数的相关内容相互印证
% 	  \end{itemize}
	\end{enumerate}
	
	\pause\pause\bigskip\hrule\bigskip
	\centerline{\ba{注意和向量值函数等相关内容的前后印证}}
\end{frame}

\begin{frame}
	\linespread{1.2}
	\begin{exampleblock}{{\bf (11)\;填空}\hfill (+4)}
		设$F(x,y)=\dint_0^{xy}\df{\sin t}{1+t^2}\d t$,则
		$\left.\df{\p^2F}{\p x^2}\right|_{x=0 \atop y=2}=$
		\underline{\quad\quad\quad}
	\end{exampleblock}
	\bigskip
	\begin{exampleblock}{{\bf (16)\;解答}\hfill (+9)}
		设$z=f(xy,yg(x))$,其中$f$二阶偏导数连续,$g(x)$可导且
		有极值$g(1)=1$,求$\left.\df{\p^2z}{\p x\p y}\right|_{x=1\atop y=1}$
	\end{exampleblock}
% 	\pause{\bf answer:}(11)\;\alert{$4$};\;
% 	(16)\alert{$f'_1(1,1)+f''_{11}(1,1)+f''_{12}(1,1)$}
\end{frame}

%=====================================
 
% \begin{frame}{title}
% 	\linespread{1.2}
% 	\begin{block}{{\bf title}\hfill}
% 		123
% 	\end{block}
% \end{frame}