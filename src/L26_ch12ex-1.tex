\begin{frame}{曲线积分与曲面积分}
	\linespread{1.2}
	\begin{enumerate}
	  \item 曲线和曲面积分的计算
	  \begin{itemize}
	    \item {\it 第一型:方向无关,第二型:方向敏感}
	    \item {\it 第一型:质量类应用;第二型:向量场类应用}
	  \end{itemize}
	  \item Green公式和Gauss公式
	  {\it
	  \begin{itemize}
	    \item Green公式的证明
	    \item “补全”和“挖洞”
	    \item 全微分与原函数(势函数)
	    \item 散度、旋度、无源场、无旋场、保守场、C-R条件
	  \end{itemize}
	  }
	  \item 对称性在各种积分中的应用
	  \item 不同类型积分之间的相互转换
	\end{enumerate}
\end{frame}

\begin{frame}
	\linespread{1.5}
% 	$\star$
	\ba{1、}设$L$为曲线$x=\df{3at}{1+t^3},y=\df{3at^2}{1+t^3}$上$t$由
	$0$到$+\infty$的一段,$a>0$,则$\dint_Lx\d y-y\d x=$
	\underline{\uncover<2->{\;\b{$3a^2$}}\;}.\\[1em]
	
% 	\pause\pause
	\ba{2、}$f(x)$连续可导,$L$为$(3,2/3)$到$(1,2)$的直线,
	则$\dint_L\df{1+y^2f(xy)}y\d x+\df x{y^2}[y^2f(xy)-1]\d y=$
	\underline{\uncover<3->{\;\b{$-4$}}\;}.\\[1em]
	
% 	\pause\pause
	\ba{3、}$\ds\iint_{z=\sqrt{a^2-x^2-y^2}}(x+y+z)\d S=$
	\underline{\uncover<4->{\;\b{$\pi a^3$}}\;}.\\[1em]
\end{frame}

\begin{frame}
	\linespread{1.5}
% 	$\star$
	\ba{4、}设$\Sigma$为平面$x+y+z=1$在第一卦限的上侧,$f(x,y,z)$连续,则
	$\ds\iint_{\Sigma}[f(x,y,z)+x]\d y\d z-[2f(x,y,z)-y]\d z\d x+
	[f(x,y,z)+z]\d x\d y=$
	\underline{\uncover<2->{\;\b{$\df12$}}\;}.\\[1em]
	
% 	\pause\pause
	\ba{5、}设$\Sigma$为锥面$z=\sqrt{x^2+y^2}(0\leq z\leq 1)$的下侧,则
	$\ds\iint_{\Sigma}x\d y\d z+2y\d z\d x+3(z-1)\d x\d y=$
	\underline{\uncover<3->{\;\b{$2\pi$}}\;}.\\[1em]
	
% 	\pause\pause
	\ba{6、}设$L$是摆线$x=t-\sin t-\pi,y=1-\cos t$从$t=0$
	到$t=2\pi$的一段,则$\dint_L\df{(x-y)\d x+(x+y)\d y}{x^2+y^2}=$
	\underline{\uncover<4->{\;\b{$\pi$}}\;}.\\[1em]
\end{frame}

\begin{frame}
	\linespread{1.5}
	\ba{例:}设$L_1:\df{x^2}{4}+\df{y^2}{9}=1,L_2:
	\df{x^2}{9}+\df{y^2}{4}=1$,
	二者所围封闭区域分别为$D_1,D_2$,则下列正确的是
	(\underline{\uncover<2->{\;\b{C}}\;})
	\begin{enumerate}[(A)]
	  \item $\dint_{L_1}(x+y^2)\d s=2\dint_{L_2}y^2\d s$
	  \item $\dint_{L_1}(x^2+y)\d s=2\dint_{L_2}(x^2+y)\d s$
	  \item $\ds\iint_{D_1}(x+y^3)\d\sigma=2\ds\iint_{D_2}(x+y^3)\d\sigma$
	  \item $\ds\iint_{D_1}(x^2+y)\d\sigma=2\ds\iint_{D_2}(x^2+y)\d\sigma$
	\end{enumerate}
\end{frame}

\begin{frame}
	\linespread{1.5}
	\ba{例:}$f(x,y)$偏导连续,曲线$L:f(x,y)=1$过第二象限的点$M$
	  和第四象限的点$N$,$\Gamma$为$L$上从$M$到$N$的一段弧,则下列
	  小于零的是
	(\underline{\uncover<2->{\;\b{B}}\;})
	\begin{enumerate}[(A)]
	  \item $\dint_{\Gamma}f(x,y)\d x$
	  \item $\dint_{\Gamma}f(x,y)\d y$
	  \item $\ds\int_{\Gamma}f(x,y)\d s$
	  \item $\ds\int_{\Gamma}f\,'_x(x,y)\d x+f\,'_y(x,y)\d y$
	\end{enumerate}
\end{frame}

\begin{frame}
	\linespread{1.5}
	\ba{例:}设曲面$S_1:x^2+y^2+z^2=1(z\geq
	  0)$,$S_2$为$S_1$在第一卦限中的部分,
	  则以下正确的是
	(\underline{\uncover<2->{\;\b{C}}\;})
	\begin{enumerate}[(A)]
	  \item $\ds\iint_{S_1}x\d S=4\iint_{S_2}x\d S$
	  \item $\ds\iint_{S_1}y\d S=4\iint_{S_2}x\d S$
	  \item $\ds\iint_{S_1}z\d S=4\iint_{S_2}x\d S$
	  \item $\ds\iint_{S_1}xyz\d S=4\iint_{S_2}xyz\d S$
	\end{enumerate}
\end{frame}

\begin{frame}
	\linespread{1.5}
	\ba{例:}设$f(r)$二阶连续可微,$r=\sqrt{x^2+y^2+z^2}$,
  	若$\mathrm{div}(\bigtriangledown\,f(r))=0$,则$f(r)=$
	(\underline{\uncover<2->{\;\b{B}}\;})
	\begin{enumerate}[(A)]
	  \item $C_1r+C_2$
      \item $C_1/r+C_2$
      \item $C_1r^2+C_2$
      \item $C_1/r^2+C_2$
	\end{enumerate}
\end{frame}

\begin{frame}
	\linespread{1.2}
	\alert{提示:}{\it\b 几个建议单独记忆的公式 
	\begin{enumerate}[(1)]
	  \item \b$\mathrm{div}(\bm{v})=\bigtriangledown\cdot\bm{v}$
	  \item \b$\mathrm{rot}(\bm{v})=\bigtriangledown\times\bm{v}$
	  \item \b$\mathrm{div}(u\;\bm{v})=u\;\mathrm{div}\bm{v}
	  +\bm{v}\cdot\bigtriangledown u$
	  \item \b$\mathrm{rot}(u\;\bm{v})=u\;\mathrm{rot}\bm{v}
	  +\bm{v}\times\bigtriangledown u$
	  \item \b$\mathrm{div}(\bigtriangledown r)=\df2r$,
	  \hspace{1cm} $(r=\sqrt{x^2+y^2+z^2})$
	  \item \b$\mathrm{rot}(\bigtriangledown r)=0$,
	  \hspace{1cm} $(r=\sqrt{x^2+y^2+z^2})$
	\end{enumerate}
	}
\end{frame}

\begin{frame}
	\linespread{1.2}
	\ba{例:}设$f(x)$当$x>0$时可导,$f(1)=2$,对右半平面内的任意封闭曲线$C$,
	有$\ds\oint_C4x^3y\d x+xf(x)\d y=0$
	\begin{enumerate}[(1)]
	  \item 求$f(x)$;
	  \item 设$L$为从$(1,0)$到$(2,3)$的一段弧,计算
	  $$\dint_L4x^3y\d x+xf(x)\d y$$
	\end{enumerate}
		
	\bigskip\pause
	\alert{提示:}{\it\b 
	由C-R条件(积分与路径无关),可解得$$f(x)=\df{x^4+1}x$$}
\end{frame}

\begin{frame}
	\linespread{1.2}
	\ba{例:}函数$u(x,y),v(x,y)$在单位圆内存在一阶连续偏导数,
	$$\bm{f}(x,y)=(v(x,y),u(x,y)),$$
	$$\bm{g}(x,y)=\left(u'_x-u'_y,v'_x-v'_y\right),$$
	在单位圆上,$u(x,y)=x,v(x,y)=1$,求
	$$\iint_{x^2+y^2\leq 1}\bm{f}\cdot\bm{g}\d\sigma$$
		
% 	\bigskip\pause
% 	\alert{提示:}{\it\b 
% 	由C-R条件(积分与路径无关),可解得$f(x)=\df{x^4+1}x$}
\end{frame}

\begin{frame}
	\linespread{1.2}
% 	\ba{例:}函数$u(x,y),v(x,y)$在单位圆内存在一阶连续偏导数,
% 	$$\bm{f}(x,y)=(v(x,y),u(x,y)),$$
% 	$$\bm{g}(x,y)=\left(u'_x-u'_y,v'_x-v'_y\right),$$
% 	在单位圆上,$u(x,y)=x,v(x,y)=1$,求
% 	$$\iint_{x^2+y^2\leq 1}\bm{f}\cdot\bm{g}\d\sigma$$
		
% 	\bigskip\pause
	\alert{提示:}
	{\it\b
	\begin{align}
		&\iint_{x^2+y^2\leq 1}\bm{f}\cdot\bm{g}\d\sigma\notag\\
		&=\iint_{x^2+y^2\leq 1}\left[(vu'_x+uv'_x)-
		(vu'_y+uv'_y)\right]\d\sigma\notag\\
		&=\iint_{x^2+y^2\leq 1}\left[(uv)'_x-(uv)'_y\right]
		\d\sigma\notag\\
		&=\oint_{x^2+y^2=1}(uv)\d x-(uv)\d y
		=\oint_{x^2+y^2=1}x\d x-x\d y\notag\\
		&=\iint_{x^2+y^2\leq 1}(-1)\d\sigma=-\pi\notag\\
	\end{align}
	}
\end{frame}

\begin{frame}
	\linespread{1.2}
	\ba{例:}设$\Sigma$为曲面$z=\sqrt{x^2+y^2}$及平面$z=1$和$z=2$
	所围立体的外表面,求
	$$\oiint_{\Sigma}\sqrt{x^2+y^2}e^z(\d y\d z+\d z\d x+\d x\d y)$$
		
	\bigskip\pause	
	\alert{思路1:}{\it\b Gauss公式,难度系数:
	\FiveStar\FiveStar\FiveStar\FiveStar\FiveStar\pause}
	
	\alert{思路2:}{\it\b 直接计算,难度系数:
	\FiveStar\FiveStar\FiveStar\FiveStar\FiveStarOpen\pause}
	
	\alert{思路3:}{\it\b 化成第一型曲面积分,难度系数:
	\FiveStar\FiveStar\FiveStar\FiveStarOpen\FiveStarOpen}
\end{frame}

\begin{frame}
	\linespread{1.2}
	\ba{思考:}找出以下推导中存在的问题
	$\Omega:r\leq R$,$\Sigma=\p\Omega$,取外侧,
	$r=\sqrt{x^2+y^2+z^2}$
	\begin{enumerate}[(1)]
	  \item
	  $\ds\oiint_{\Sigma}\df{x^3}{r^3}\d
	  y\d z+\df{y^3}{r^3}\d z\d x+\df{z^3}{r^3}\d x\d y$\\ 
	  \hspace{2cm}$=\df 1{R^3}\oiint_{\Sigma}x^3\d y\d z+y^3\d z\d x+z^3\d x\d y$\\ 
	  \hspace{2cm}$=\df 1{R^3}\iiint_{\Omega}3r^2\d V{=\df
	  3R\iiint_{\Omega}\d V} =4\pi R^2$\pause
	  \item
	  $\ds\oiint_{\Sigma}\df{x^3}{r^3}\d
	  y\d z+\df{y^3}{r^3}\d z\d x+\df{z^3}{r^3}\d x\d y$\\
	  \hspace{2cm}$=\ds\iiint_{\Omega}\left[\df{\p}{\p x}\df{x^3}{r^3}+\df{\p}{\p
	  y}\df{y^3}{r^3}+\df{\p}{\p
	  z}\df{z^3}{r^3}\right]\d V$
	\end{enumerate}
\end{frame}

\begin{frame}
	\linespread{1.2}
	\ba{例:}设$\Sigma$为$x^2+y^2=R^2$及平面$z=\pm R\,(R>0)$所围立体的外表面,求
	$$\oiint_{\Sigma}\df{x\d y\d z+y^2\d z\d x+z^2\d x\d y}{x^2+y^2+z^2}$$
		
	\bigskip\pause
	\alert{提示:}{\it\b
	利用第二型曲面积分的对称性,原式可化为
	$$\oiint_{\Sigma}\df{x\d y\d z}{x^2+y^2+z^2}$$
	}
\end{frame}

\begin{frame}
	\linespread{1.2}
	\ba{例:}设$\Sigma$为$2x^2+2y^2+z^2=4$的外侧,求
	$$\oiint_{\Sigma}\df{x\d y\d z+y\d z\d x+z\d x\d y}{(x^2+y^2+z^2)^{3/2}}$$
		
	\bigskip\pause
	\alert{提示:}{\it\b
	可以验证,在原点之外旋度为零,“挖洞”后使用Gauss公式!\pause
	}
	
	\alert{思考:}{\it\b如何计算
	$$\oiint_{\Sigma}\df{x\d y\d z+y\d z\d x+z\d x\d y}
	{(x^2+2y^2+2z^2)^{3/2}}$$
	}
	
	\pause\alert{提示:}{\it\b
	挖一个椭圆形的“洞”:$x^2+2y^2+2z^2=\e$
	}
\end{frame}

\begin{frame}
	\linespread{1.2}
	\ba{例:}在变力$\bm{F}=(yz,zx,xy)$的作用下,质点由原点沿直线运动到椭球面
	$\df{x^2}{a^2}+\df{y^2}{b^2}+\df{z^2}{c^2}=1$上第一卦限
	中的某点$M$,问$M$在何位置时,$\bm{F}$所做的功最大,并求出功的最大值。
		
	\bigskip\pause
	\alert{提示:}{\it\b
	首先验证$\bm{F}$为保守场,再根据$M$在椭球面的位置,求功的最大值
	}
\end{frame}