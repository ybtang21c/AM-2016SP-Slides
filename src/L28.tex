% !Mode:: "TeX:UTF-8"

\begin{frame}{第二十八讲、幂级数及其应用}
	\linespread{1.5}
	\begin{enumerate}
	  \item {\bf 内容与要求}{\b (\S 13.1)}
	  \begin{itemize}
		\item 理解函数项级数的概念
		\item 熟练掌握幂级数收敛域的计算方法
		\item 熟练掌握幂级数的和函数计算方法
		\item 熟练掌握函数的幂级数展开
		\item 了解幂级数的应用
	  \vspace{1em}
	  \end{itemize}
	  \item {\bf  课后作业:}
	  \begin{itemize}
	    \item {\b 习题13.1:4(3,6,8),6,7,16}
	  \end{itemize}
	\end{enumerate}
\end{frame}

\section{函数项级数的概念}

\begin{frame}{数项级数与函数项级数}
	\linespread{1.5}
	\begin{enumerate}\pause 
	  \item {\bb 数项级数(无穷级数):}\pause 无穷多个\alert{数值}依序相加\pause 
	  $$S=\sum\limits_{n=1}^{\infty}u_n
	  \pause =\lim\limits_{n\to\infty}\sum\limits_{k=1}^nu_k$$\pause 
	  \item {\bb 函数项级数:}\pause 无穷多个\alert{函数}依序相加\pause 
	  $$S(x)=\sum\limits_{n=1}^{\infty}u_n(x)
	  =\lim\limits_{n\to\infty}\sum\limits_{k=1}^nu_k(x)$$
	\end{enumerate}
\end{frame}

\begin{frame}{函数项级数}
	\linespread{1.2}
	$$\alert{S(x)=\sum\limits_{n=0}^{\infty}u_n(x)}$$
	\pause 
	\begin{enumerate}
	  \item {\bb 幂级数:}\pause 
	  $$S(x)=\sum\limits_{n=0}^{\infty}a_n(x-x_0)^n$$\pause 
	  \item {\bb Fourier级数:}\pause 
	  $$S(x)=\df{a_0}{2}+\sum\limits_{n=1}^{\infty}
	  (a_n\cos nx+b_n\sin nx)$$
	\end{enumerate}
\end{frame}

\section{幂级数的概念与性质}

\begin{frame}{幂级数}
	\linespread{1.2}
	$$\alert{S(x)=\sum\limits_{n=0}^{\infty}a_n(x-x_0)^n}$$
	其中:$a_n\in\mathbb{R}\,(n\in\mathbb{N})$,$x,x_0\in\mathbb{R}$。
	
	\bigskip
	\pause \quad 令\,{$t=x-x_0$},\pause 以下重点研究
	$$\alert{S(x)=\sum\limits_{n=0}^{\infty}a_nx^n}$$
\end{frame}

\begin{frame}{1、幂级数的收敛性}
	\linespread{1.8}
	\begin{exampleblock}{{\bf 例1:}讨论下列幂级数的收敛性\hfill}
		\begin{columns}\pause 
			\column{.4\textwidth}
				\begin{enumerate}
				  \item $\sum\limits_{n=0}^{\infty}\df{x^n}{n!}$\pause 
				  \item $\sum\limits_{n=0}^{\infty}x^n$\pause 
				  \item $\sum\limits_{n=0}^{\infty}n!x^n$\pause 
				\end{enumerate}
			\column{.6\textwidth}
				\alert{$(-\infty<x<+\infty)$}\pause \\[4pt]
				\alert{$(|x|<1)$}\pause \\[4pt]
				\alert{($x=0)$}\pause 
		\end{columns}
	\end{exampleblock}
	\vspace{-1em}
	\begin{itemize}
	  \item \alert{幂级数的收敛域的特点:\pause {\bf 非空、\pause 对称}}
	\end{itemize}
\end{frame}

\begin{frame}
	\linespread{1.2}\pause 
	\begin{block}{{\bf 定理13.1.1}(Abel定理)\hfill}
		考虑幂级数
		$$\sum\limits_{n=0}^{\infty}a_nx^n$$
		\pause
% 		\vspace{-1em}
		\begin{enumerate}
		  \item 若在$x_0\ne 0$处该级数收敛,则对任意$|x|<|x_0|$,该级数绝对收敛;\pause 
		  \item 若在$x_0\ne 0$处该级数发散,则对任意$|x|>|x_0|$,该级数发散。\pause 
		\end{enumerate}
	\end{block}
	\begin{itemize}
% 	  \item \alert{{\bf 注:}幂级数在其收敛域内绝对收敛}\pause
	  \item \alert{{\bf 推论:}存在幂级数收敛与发散的边界点:$x^*=\pm R$}
	\end{itemize}
\end{frame}

\begin{frame}{幂级数的收敛域}
	\linespread{1.2}
	$$\sum\limits_{n=0}^{\infty}a_n(x-x_0)^n$$
	\pause 在以$x_0$为中心$R$({\bb 收敛半径})为半径的对称区间内收敛\pause 
	
% 	\ba{问:}
	\begin{enumerate}
	  \item 在区域的边界处,幂级数的收敛性如何?\pause 
	  
	  \alert{利用数项级数的收敛性判别法单独判定}\pause
	  \item 若级数在$x_1$处条件收敛,$R=$
	  \pause\underline{\alert{\;$|x_1-x_0|$\;}}\pause
	\end{enumerate}
% 	\begin{center}
% 		\ba{已知幂级数,如何求其收敛域?}
% 	\end{center}
\end{frame}

\begin{frame}
	\linespread{1.2}
	\begin{block}{{\bf 定理13.1.3-4}\hfill}
		若$\sum\limits_{n=0}^{\infty}a_nx^n$的系数满足
		$$\limn\left|\df{a_{n+1}}{a_n}\right|=\rho
		\quad\mbox{或}\quad\limn\sqrt[n]{|a_n|}=\rho,$$
		则$R=1/\rho$。
	\end{block}
	\pause 
	\begin{exampleblock}{{\bf 例2:}求下列幂级数的收敛域\hfill}
		\begin{columns}
			\column{.4\textwidth}
				\begin{enumerate}
				  \item $\sum\limits_{n=0}^{\infty}\df{2+(-1)^n}{2^n}x^n$\pause 
				\end{enumerate}
			\column{.32\textwidth}
				\begin{enumerate}
				  \addtocounter{enumi}{1}
				  \item $\sum\limits_{n=0}^{\infty}\df{(x-1)^n}{2^nn}$\pause 
				\end{enumerate}
			\column{.28\textwidth}
				\begin{enumerate}
				  \addtocounter{enumi}{2}
				  \item $\sum\limits_{n=0}^{\infty}2^nx^{2n}$
				\end{enumerate}
		\end{columns}
	\end{exampleblock}
\end{frame}

\begin{frame}{求幂级数的收敛域}
	\linespread{1.2}\pause
	\begin{center}
		{\bb 先求半径,再讨论端点}
		\vspace{1ex}\hrule
		\pause 
	\end{center}
	\vspace{-1ex}
	\begin{enumerate}
	  \item 标准型幂级数:\pause $\sum\limits_{n=0}^{\infty}a_nx^n$\pause 
	  $$\alert{R=\limn\left|\df{a_n}{a_{n+1}}\right|}
	  \pause\quad\mbox{或}\quad \alert{R=\df 1{\limn\sqrt[n]{|a_n|}}}$$\pause 
	  \vspace{-1em}
	  \item 非标准型幂级数:\pause 缺项或通项为复合函数\pause 
	  \begin{center}
	  	\alert{\underline{先换元、整理,再求收敛半径}}\pause 
	  \end{center}
	  \item \alert{比值法极限不存在时,注意使用根值法}
	\end{enumerate}
\end{frame}

\begin{frame}{2、幂级数的和函数}
	\linespread{1.4}\pause 
	\begin{block}{{\bf 定理3}(幂级数的运算性质)\hfill}\pause 
		\begin{enumerate}
		  \item 两个幂级数的和在其收敛域的交集内收敛\pause 
% 		  
% 		  \alert{(求和后幂级数的收敛半径不小于原幂级数收敛半径的最小值\pause
% 		  ,{\bf 例:}$\sum x^n-\sum x^n/(1+a^n),(0<a<1)$)}\pause
		  \item 幂级数逐项求导或求不定积分后收敛半径不变;\pause 
		  \item 幂函数的和函数在其收敛域上连续;\pause 
		  \item 幂函数逐项求导或求不定积分后,对应和函数分别为原和函数的导函数和原函数。
		\end{enumerate}
	\end{block}
\end{frame}

\begin{frame}
	\linespread{1.2}
	\begin{exampleblock}{{\bf 例3}\hfill}
		已知
		$$\df{1}{x+1}=\sum\limits_{n=0}^{\infty}(-x)^n\;(|x|<1),$$
		求幂级数
		$$\sum\limits_{n=1}^{\infty}\df{(-1)^{n-1}}{n}x^n$$
		的和函数。由此证明:
		$$\ln 2=1-\df 12+\df 13-\df 14+\ldots+\df{(-1)^n}{n}+\ldots$$
	\end{exampleblock}
\end{frame}

\begin{frame}
	\linespread{1.2}
	\begin{exampleblock}{{\bf 例4:}求下列幂级数的和函数\hfill}
		\begin{columns}
			\column{.5\textwidth}
				\begin{enumerate}
				  \item $\sum\limits_{n=0}^{\infty}\df{x^n}{n!}$
				  \item $\sum\limits_{n=0}^{\infty}nx^n$
				\end{enumerate}
			\column{.5\textwidth}
				\begin{enumerate}
				  \addtocounter{enumi}{2}
				  \item $\sum\limits_{n=0}^{\infty}\df{x^n}{n+1}$
				  \item $\sum\limits_{n=0}^{\infty}n^2x^n$
				\end{enumerate}
		\end{columns}
	\end{exampleblock}
	\pause\bigskip
	\begin{exampleblock}{{\bf 例5}\hfill}
		求数项级数$\sumn\sum\limits_{k=1}^n\df{k}{2^{n-1}}$的和。
	\end{exampleblock}
\end{frame}

\begin{frame}{3、函数的幂级数展开}
	\linespread{1.2}
	函数$f(x)$在原点处的无穷阶Taylor展开式:\pause 
	$$f(x)=\sum\limits_{n=0}^{\infty}\df{f^{(n)}(0)}{n!}x^n$$
	\pause 称为:{\bb $f(x)$在原点处的Taylor级数(Maclaurin级数)}\pause 
	\begin{itemize}
	  \item \alert{{\bf 展开的条件:}\pause $f(x)$在原点处无穷阶可导}
	\end{itemize}
\end{frame}

\begin{frame}
	\linespread{1.5}
	\begin{exampleblock}{{\bf 例6:}求如下函数的Maclaurin级数\hfill}\pause 
		\begin{enumerate}
		  \item $f(x)=1+x+4x^3-5x^{99}+x^{100}$\pause 
		  \item $f(x)=(1-x)e^x$\pause 
		  \item $f(x)=\sin(x-\pi/4)$\pause 
		  \item $f(x)=\df 1{1-x^2}$\pause 
		  \item $f(x)=\ln(2+x)$\pause 
		  \item $f(x)=\arctan x$
		\end{enumerate}
	\end{exampleblock}
\end{frame}

\section{幂级数的应用}

\begin{frame}{幂级数的应用}
	\linespread{1.2}\pause 
	\begin{enumerate}
	  \item {\bf 函数值的近似计算}\pause 
	  \item {\bf 积分的近似计算:}\pause $\dint_0^1e^{-x^2}dx$\pause 
	  \item {\bf 解微分方程:}\pause 
	\end{enumerate}
	\begin{exampleblock}{{\bf 例7}\hfill}
		求解初值问题:
		$$xy''+y'+xy=0,\;y(0)=1,\;y'(0)=0$$
	\end{exampleblock}
\end{frame}

\begin{frame}[<+->]{小结}
	\linespread{1.2}
	\begin{enumerate}
	  \item {\bf 函数项级数的概念}
	  \item {\bf 幂级数:}
	  $$S(x)=\sum\limits_{n=1}^{\infty}a_n(x-x_0)^n$$
	  \begin{itemize}
	    \item 收敛域与收敛半径
	    \item 幂级数的和函数
	    \item 函数的幂级数(Taylor)展开
	  \end{itemize}
	  \item {\bf 幂级数的应用}
	\end{enumerate}
\end{frame}

\begin{frame}{课堂练习}
	\linespread{1.5}
	\begin{exampleblock}{{\bf 例8:}判断正误\hfill}
		\begin{enumerate}
		  \item 幂级数在其收敛域上一定绝对收敛\pause \;{\alert{$(\;\times\;)$}}\pause 
		  \item 设$\sum a_nx^n,\sum b_nx^n$的收敛半径分别为$R_1,R_2$,
		  则$\sum(a_n+b_n)x^n$的收敛半径$R=\min\{R_1,R_2\}$\pause 
		  \;{\alert{$(\;\times\;)$}}\pause 
		  \item 幂级数与其逐项积分和逐项求导的幂级数的收敛域相同\pause 
		  \;{\alert{$(\;\times\;)$}}
		\end{enumerate}
	\end{exampleblock}
\end{frame}

\begin{frame}
	\linespread{1.5}
	\begin{exampleblock}{{\bf 例9}\hfill}
		若$\sum a_n(x-2)^n$在$x=-2$收敛,则在$x=5$处
		\underline{\;\uncover<2->{\alert{绝对收敛}}\;};若在$x=-2$处条件收敛,
		则收敛半径为\underline{\;\uncover<3->{\alert{$4$}}\;}
	\end{exampleblock}
	\onslide<4->
	\bigskip
	\begin{exampleblock}{{\bf 例10}\hfill}
		求函数项级数$\sum\limits_{n=1}^{\infty}
		\sin\df 1 {3n}\left(\df{3+x}{3-2x}\right)^n$
		的收敛域。
	\end{exampleblock}
\end{frame}

\begin{frame}
	\linespread{1.2}
	\begin{exampleblock}{{\bf 例11}\hfill}
		求级数$\sum\limits_{n=1}^{\infty}\df{n}{2^{n-1}}$的和。
	\end{exampleblock}
	\bigskip
	\pause
	\begin{exampleblock}{{\bf 例12}\hfill}
		将下列函数展开成Maclaurin级数,并求其收敛域
		\begin{enumerate}
		  \item $f(x)=\arcsin x$
		  \item $f(x)=\df 1{1+x+x^2}$
		\end{enumerate}
	\end{exampleblock}
\end{frame}

%=====================================

% \begin{frame}{title}
% 	\linespread{1.2}
% 	\begin{exampleblock}{{\bf title}\hfill}
% 		123
% 	\end{exampleblock}
% \end{frame}
% 
% \begin{frame}{title}
% 	\linespread{1.2}
% 	\begin{block}{{\bf title}\hfill}
% 		123
% 	\end{block}
% \end{frame}