% !Mode:: "TeX:UTF-8"

\begin{frame}{第十三讲、多元函数的偏导数习题课}
	\linespread{1.5}
	\begin{enumerate}
	  \item {\bf 内容与要求}
	  \begin{itemize}
	    \item 多元函数的极限与连续性
	    \item 多元复合函数微分的链式法则
	    \item 一阶微分形式的不变性
	    \item 隐函数的微分法
	  \vspace{1em}
	  \end{itemize}
% 	  \item {\bf  课后作业:}
% 	  \begin{itemize}
% 	    \item {\b 习题10.3:7,9,12,14,16}
% 	  \end{itemize}
	\end{enumerate}
\end{frame}

\begin{frame}{多元函数的极限与连续}
	\linespread{1.5}
	\begin{exampleblock}{{\bf 例0}\hfill}
		  证明函数
		  $$f(x,y)=\left\{\begin{array}{cc}
		  	\df{x^2y}{x^2+y^2},& (x,y)\ne (0,0)\\
		  	0,& else
		  \end{array}\right.$$
		  在原点可微,但偏导数不连续。
	\end{exampleblock}
\end{frame}

\begin{frame}{复合函数的偏导数}
	\linespread{1.5}
	\begin{exampleblock}{{\bf 例1}\hfill}
		\begin{enumerate}
		  \item 设$z=u^2\ln v,u=\df xy,v=3x-2y$,求$\df {\p z}{\p x},\df{\p z}{\p
		  y}$\pause 
		  \item 设$z=\arctan(xy),y=e^x$,求$\df{\d z}{\d x}$\pause 
		  \item 设$u=f\left(\df xy,\df yz\right)$,求$\df {\p u}{\p x},\df {\p u}{\p y}$
		  和$\df {\p u}{\p z}$\pause 
		  \item 设$f(x,2x)=x,f\,'_1(x,2x)=x^2$,求$f\,'_2(x,2x)$
		\end{enumerate}
	\end{exampleblock}
\end{frame}

\begin{frame}
	\linespread{1.2}
	\begin{exampleblock}{{\bf 例2}\hfill}
		设$f(x,y)$一阶偏导连续,$f(1,1)=1,f\,'_x(1,1)=2$,
		
		$f\,'_y(1,1)=3$,又$\varphi(x)=f(x,f(x,x))$,求
		$$\left.\df{\d\varphi^3(x)}{\d x}\right|_{x=1}$$
	\end{exampleblock}
\end{frame}

\begin{frame}
	\linespread{1.2}
	\begin{exampleblock}{{\bf 例3}\hfill}
		设$z=xf\left(\df yx\right)+yg\left(x,\df xy\right)$,
		其中$f,g$均二次可微,求
		$$\df{\p^2 z}{\p x\p y}$$
	\end{exampleblock}
\end{frame}

\begin{frame}{隐函数求导}
	\linespread{1.2}
	\begin{exampleblock}{{\bf 例4}\hfill}
		设$z=z(x,y)$由方程$x^2+y^2+z^2=xf\left(\df yx\right)$确定,
		且$f$可微,求$\df {\p z}{\p x},\df{\p z}{\p y}$
	\end{exampleblock}\pause 
	\begin{exampleblock}{{\bf 例5}\hfill}
		设$y=f(x,t)$,其中$t$是由$F(x,y,t)=0$所确定的隐函数,$f,F$具有
		一阶连续偏导数,求$\df{\d y}{\d x}$
	\end{exampleblock}
\end{frame}

\begin{frame}
	\linespread{1.2}
	\begin{exampleblock}{{\bf 例6}\hfill}
		设$F(u,v)$具有一阶连续偏导数,且由$F\left(\df xz,\df yz\right)=0$
		可确定函数$z=z(x,y)$,求
		$$x\df{\p z}{\p x}+y\df{\p z}{\p y}$$
	\end{exampleblock}
\end{frame}

\begin{frame}
	\linespread{1.2}
	\begin{exampleblock}{{\bf 例7}\hfill}
		设$u=u(x,y),x=r\cos\theta,y=r\sin\theta$,将方程
		$$x\df{\p u}{\p y}-y\df{\p u}{\p x}=0$$
		化为$u$关于$r,\theta$的偏导数的方程。
	\end{exampleblock}
\end{frame}

\begin{frame}
	\linespread{1.2}
	\begin{exampleblock}{{\bf 例8}\hfill}
		设$u=u(x,y,z)$具有连续偏导数,且
		$$x=r\sin\theta\cos\varphi,y=r\sin\theta\sin\varphi,z=r\cos\theta$$
		证明:若$x\df{\p u}{\p x}+y\df{\p u}{\p y}+z\df{\p u}{\p z}=0$,
		则$u$与$r$无关。
	\end{exampleblock}
\end{frame}

\begin{frame}
	\linespread{1.2}
	\begin{exampleblock}{{\bf 例9}\hfill}
		设$f(x,y)$一阶偏导连续,$f(1,1)=1,f\,'_1(1,1)=a$,
		
		$f\,'_2(1,1)=b$,又$\varphi(x)=f(x,f(x,f(x,x)))$,求
		$\varphi(1)$与$\varphi'(1)$。
	\end{exampleblock}
	\bigskip\pause 
	\begin{exampleblock}{{\bf 例10}\hfill}
		设由$\ln(xz)+\arctan(yz)=0$可确定隐函数$z=z(x,y)$,求$z'_x$。
	\end{exampleblock}
\end{frame}

%=====================================
 
% \begin{frame}{title}
% 	\linespread{1.2}
% 	\begin{exampleblock}{{\bf title}\hfill}
% 		123
% 	\end{exampleblock}
% \end{frame}

% \begin{frame}{title}
% 	\linespread{1.2}
% 	\begin{block}{{\bf title}\hfill}
% 		123
% 	\end{block}
% \end{frame}