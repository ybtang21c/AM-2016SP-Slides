% !Mode:: "TeX:UTF-8"

\begin{frame}{第二十七讲、对称性在积分计算中的应用}
	\linespread{1.5}
	\begin{enumerate}
	  \item {\bf 关键词}
	  \begin{itemize}
		\item (函数的)奇偶性
		\item (区域的)对称性
		\item (符号的)对称性
		\item 定积分、重积分、曲线积分、曲面积分
	  \vspace{1em}
	  \end{itemize}
% 	  \item {\bf  课后作业:}
% 	  \begin{itemize}
% 	    \item {\b 习题12.4:1,2,5,6,8,14}
% 	  \end{itemize}
	\end{enumerate}
\end{frame}

% \section{定积分中的对称性}
% 
% \begin{frame}{定积分中的对称性}
% 	\linespread{1.2}
% 	\begin{block}{{\bf 命题1.1}\hfill}
% 		$$\dint_{-a}^af(x)dx=\left\{\begin{array}{ll}
% 			0,& f(x)\mbox{是奇函数}\\
% 			2\dint_0^af(x)dx,& f(x)\mbox{是偶函数}
% 		\end{array}\right.$$
% 	\end{block}
% \end{frame}
% 
% \begin{frame}{定积分中的对称性}
% 	\linespread{1.2}
% 	\begin{block}{{\bf 命题1.2}\hfill}
% 		$$\dint_0^{\pi/2}f(\sin x)dx=\dint_0^{\pi/2}f(\cos x)dx$$
% 	\end{block}
% 	\begin{exampleblock}{{\bf 例1}\hfill}
% 		$$\dint_0^{\pi/2}\df{\sin t}{\sin t+\cos t}dt$$
% 	\end{exampleblock}
% \end{frame}

% \section{二重积分中的对称性}

\begin{frame}{二重积分中的对称性}
	\linespread{1.2}
	\begin{block}{{\bf 命题2.1}(坐标对称)\hfill}
		区域$D$关于$x=0$对称,$D_1$为其中$x\geq 0$的部分,则
		$$\iint_Df(x,y)d\sigma=\left\{\begin{array}{ll}
			0,& f\mbox{关于}x\mbox{是奇函数}\\
			2\ds\iint_{D_1}f(x,y)d\sigma,& f\mbox{关于}x\mbox{是偶函数}
		\end{array}\right.$$
	\end{block}
	\pause
	\alert{{\bf 注:}将以上关于$x$的性质改成关于$y$的,结论同样成立}
\end{frame}

\begin{frame}{二重积分中的对称性}
	\linespread{1.2}
	\begin{block}{{\bf 命题2.2}\hfill}
		区域$D$关于$x=0,y=0$都对称,$D_1$为其在第一象限内的部分,
		若$f(x,y)$关于$x,y$均为偶函数,则
		$$\iint_Df(x,y)d\sigma=4\iint_{D_1}f(x,y)d\sigma$$
	\end{block}
	\pause
	\begin{exampleblock}{{\bf 例1}\hfill}
		$$\iint_{x^2+y^2\leq 1}\df{(x+2y)^2}{1+x^2+y^2}d\sigma$$
	\end{exampleblock}
\end{frame}

\begin{frame}{二重积分中的对称性}
	\linespread{1.2}
	\begin{block}{{\bf 命题2.3}(轴对称)\hfill}
		区域$D$关于某条直线对称,对$D$内关于该轴对称的任意两点$M,N$,总有
		$f(M)=-f(N)$,则
		$$\iint_Df(x,y)d\sigma=0$$
	\end{block}
	\pause
	\begin{exampleblock}{{\bf 例2}\hfill}
	$D:0\leq x\leq 1,0\leq y\leq 1$,求
		$$\iint_D(x+y)\mathrm{sgn}(x-y)d\sigma$$
	\end{exampleblock}
\end{frame}

\begin{frame}{二重积分中的对称性}
	\linespread{1.2}
	\begin{block}{{\bf 命题2.4}\hfill}
		区域$D$关于$y=x$对称,则
		$$\iint_Df(x,y)d\sigma=\iint_Df(y,x)d\sigma$$
	\end{block}
	\pause
	\begin{exampleblock}{{\bf 例3}\hfill}
		区域$D$为第一象限内的四分之一单位圆,求
		$$\iint_D\df{2x^2+x-y+1}{\sqrt{1-x^2-y^2}}d\sigma$$
	\end{exampleblock}
\end{frame}

\begin{frame}{二重积分中的对称性}
	\linespread{1.2}
	\begin{block}{{\bf 命题2.6}(轮换对称)\hfill}
		区域$D$内$x,y$地位相同,则
		$$\iint_Df(x,y)d\sigma=\iint_Df(y,x)d\sigma$$
	\end{block}
	\pause
	\begin{exampleblock}{{\bf 例4}\hfill}
		区域$D$为第一象限内半径为$2$的圆,$f(x)$恒为正,求
		$$\iint_D\df{a\sqrt{f(x)}+a\sqrt{f(y)}}{\sqrt{f(x)}+\sqrt{f(y)}}d\sigma$$
	\end{exampleblock}
\end{frame}

% \section{三重积分中的对称性}

\begin{frame}{三重积分中的对称性}
	\linespread{1.2}
	\begin{block}{{\bf 命题3.1}(坐标对称)\hfill}
		空间区域$\Omega$关于$z=0$对称,$\Omega_1$为其中$z\geq 0$的部分,则
		$$\iiint_{\Omega}fdV=\left\{\begin{array}{ll}
			0,& f\mbox{关于}z\mbox{是奇函数}\\
			2\ds\iiint_{\Omega_1}fdV,& f\mbox{关于}z\mbox{是偶函数}
		\end{array}\right.$$
	\end{block}
\end{frame}

\begin{frame}{三重积分中的对称性}
	\linespread{1.2}
	\begin{block}{{\bf 命题3.2}(平面对称)\hfill}
		区域$\Omega$关于某平面对称,对$\Omega$内关于该平面对称的任意两点$M,N$,
		总有$f(M)=-f(N)$,则
		$$\iiint_{\Omega}f(x,y,z)dV=0$$
	\end{block}
	\begin{block}{{\bf 命题3.3}\hfill}
		区域$\Omega$关于平面$y=x$对称,则
		$$\iiint_{\Omega}f(x,y,z)dV=\iiint_{\Omega}f(y,x,z)dV$$
	\end{block}
\end{frame}

\begin{frame}{三重积分中的对称性}
	\linespread{1.2}
	\begin{block}{{\bf 命题3.4}(轮换对称)\hfill}
		若区域$\Omega$中$x,y,z$地位相同(例:$\Omega$为一球体),则积分
		$$\iiint_{\Omega}f(x,y,z)dV$$
		中$x,y,z$的次序可以任意交换。
	\end{block}
	\pause
	\begin{exampleblock}{{\bf 例5}\hfill}
		$$\iiint_{x^2+y^2+z^2\leq R^2}\df{x^2-y^2}{(1+x^2+y^2+z^2)
		^{3/2}}dV$$
	\end{exampleblock}
\end{frame}

% \section{对弧长的曲线积分中的对称性}

\begin{frame}{对弧长的曲线积分中的对称性}
	\linespread{1.2}
	\begin{block}{{\bf 命题4.1}(轴对称)\hfill}
		曲线$L$关于$x=0$对称,$L_1$为其中$x\geq 0$的部分,则
		$$\int_{L}f(x,y)ds=\left\{\begin{array}{ll}
			0,& f\mbox{关于}x\mbox{是奇函数}\\
			2\ds\int_{L_1}fds,& f\mbox{关于}x\mbox{是偶函数}
		\end{array}\right.$$
	\end{block}
	\pause
	\begin{exampleblock}{{\bf 例6}\hfill}
		$$\oint_{|x|+|y|=1}\df{x}{|x|+|y|}ds$$
	\end{exampleblock}
\end{frame}

\begin{frame}{对弧长的曲线积分中的对称性}
	\linespread{1.2}
	\begin{block}{{\bf 命题4.2}(平面对称)\hfill}
		空间曲线$L$关于$x=0$对称,$L_1$为其中$x\geq 0$的部分,则
		$$\int_{L}f(x,y,z)ds=\left\{\begin{array}{ll}
			0,& f\mbox{关于}x\mbox{是奇函数}\\
			2\ds\int_{L_1}fds,& f\mbox{关于}x\mbox{是偶函数}
		\end{array}\right.$$
	\end{block}
\end{frame}

% \section{对面积的曲面积分中的对称性}

\begin{frame}{对面积的曲面积分中的对称性}
	\linespread{1.2}
	\begin{block}{{\bf 命题5.1}(平面对称)\hfill}
		曲线$\Sigma$关于$x=0$对称,$\Sigma_1$为其中$x\geq 0$的部分,则
		$$\iint_{\Sigma}f(x,y,z)dS=\left\{\begin{array}{ll}
			0,& f\mbox{关于}x\mbox{是奇函数}\\
			2\ds\iint_{\Sigma_1}fdS,& f\mbox{关于}x\mbox{是偶函数}
		\end{array}\right.$$
	\end{block}
	\pause
	\begin{exampleblock}{{\bf 例7}\hfill}
		$\Sigma$为$x^2+y^2=R^2$在$z=0$和$z=1$间的部分,求
		$$\iint_{\Sigma}\df y{x^2+y^2+z^2}dS$$
	\end{exampleblock}
\end{frame}

% \begin{frame}{对面积的曲面积分中的对称性}
% 	\linespread{1.2}
% 	
% \end{frame}

\begin{frame}{对面积的曲面积分中的对称性}
	\linespread{1.2}
	\begin{block}{{\bf 命题5.2}(轮换对称)\hfill}
		曲面$\Sigma$上$x,y,z$地位相同(例:$\Sigma$为一球面),则
		$$\iint_{\Sigma}f(x,y,z)dS$$
		中$x,y,z$的次序可以任意交换。
	\end{block}
	\pause
	\begin{exampleblock}{{\bf 例8}\hfill}
		$$\oint_{x^2+y^2+z^2=R^2}(x^2+3y^2)dS$$
	\end{exampleblock}
\end{frame}

% \section{对坐标的曲线、曲面积分中的对称性}

\begin{frame}{对坐标的曲线、曲面积分中的对称性}
	\linespread{1.2}
	\begin{block}{{\bf 命题6.1}(对坐标的曲面积分)\hfill}
		空间曲面$\Sigma$关于$z=0$对称,$\Sigma_1$为其中$z\geq 0$的部分,则
		$$\iint_{\Sigma}f(x,y,z)dxdy=\left\{\begin{array}{ll}
			0,& f\mbox{关于}x\mbox{是偶函数}\\
			2\ds\iint_{\Sigma_1}fdxdy,& f\mbox{关于}x\mbox{是奇函数}
		\end{array}\right.$$
	\end{block}
	\pause
	\begin{exampleblock}{{\bf 例9}\hfill}
		$$\oiint_{x^2+y^2+z^2=R^2}\df{z^2}{x^2+3y^2+1}dS$$
	\end{exampleblock}
\end{frame}

\begin{frame}{小结}
	\linespread{1.5}
	\begin{enumerate}
	  \item {\bf 轴对称性:}坐标对称、平面对称
	  \item {\bf 轮换对称性:}$x,y,z$地位对等
	\end{enumerate}
	\pause\bigskip\hrule
	\begin{center}
		\ba{对称性:积分区域的对称性}
		
		\bigskip\pause
		\ba{根据不同类型的积分灵活应用}
	\end{center}
\end{frame}

% \section{其他对称性}
% 
% \begin{frame}{其他对称性}
% 	\linespread{1.2}
% 	\begin{block}{{\bf 命题6.1}\hfill}
% 		$f(x,y)$连续,$L,L'$分别为从$(0,0)$到$(\pi/2,\pi/2)$和$(\pi,\pi)$的直线段,则
% 		\begin{enumerate}
% 		  \item $\dint_Lf(\sin x,\sin y)ds
% 		  =\dint_Lf(\cos x,\cos y)ds$
% 		  \item $\dint_Lf(\sin x,\cos y)ds
% 		  =\dint_Lf(\cos x,\sin y)ds$
% 		  \item $\dint_{L'}(x+y)f(\sin x,\sin y)ds
% 		  =\pi\dint_{L'}f(\sin x,\sin y)ds$
% 		\end{enumerate}
% 	\end{block}
% \end{frame}



%=====================================

% \begin{frame}{title}
% 	\linespread{1.2}
% 	\begin{exampleblock}{{\bf title}\hfill}
% 		123
% 	\end{exampleblock}
% \end{frame}
% 
% \begin{frame}{title}
% 	\linespread{1.2}
% 	\begin{block}{{\bf title}\hfill}
% 		123
% 	\end{block}
% \end{frame}