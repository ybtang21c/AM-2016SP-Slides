% !Mode:: "TeX:UTF-8"

\begin{frame}{第十六讲、多元函数微分的应用}
	\linespread{1.5}
	\begin{enumerate}
	  \item {\bf 内容与要求}
	  \begin{itemize}
% 	    \item 多元函数偏导数的意义
	    \item 方向导数与梯度
	    \item Hessian矩阵与Taylor公式
	    \item 多元函数的极值
	    \item 条件极值与Lagrange乘子法
	  \vspace{1em}
	  \end{itemize}
	\end{enumerate}
\end{frame}

% \begin{frame}{问题讨论}
% 	\linespread{1.2}
% % 	\ba{多元函数的方向导数与偏导数有何关系?}
% % 	\begin{enumerate}
% % 	  \item 若$f(\bm{x})$可微,则
% % 	  $$D_{\bm{}}f(\bm{x})$$
% % 	\end{enumerate}
% 	{\onslide<2->}
% 	\begin{exampleblock}{{\bf 例1:}判断正误\hfill}
% 		\begin{enumerate}
% 		  \item 若$z=f(x,y)$在某点偏导数连续,则其在该点沿任意方向的方向导数存在
% 		  \quad(\underline{\;\uncover<3->{\alert{$\surd$}}\;})
% 		  \item<4-> 若$z=f(x,y)$在某点沿任意方向的方向导数存在,则其在该点可微
% 		  \quad(\underline{\;\uncover<5->{\alert{$\times$}}\;})
% 		  \item<6-> 若$z=f(x,y)$在$(x_0,y_0)$处沿任意方向的方向导数存在,则
% 		  $f\,'_x(x_0,y_0)=D_if(x_0,y_0), \; f\,'_y(x_0,y_0)=$
% 		  
% 		  $D_jf(x_0,y_0)$ \quad(\underline{\;\uncover<7->{\alert{$\surd$}}\;})
% 		\end{enumerate}
% 	\end{exampleblock}
% % 	\begin{itemize}
% % 	  \item \alert{方向导数与偏导数之间没有直接的联系}
% % 	\end{itemize}
% \end{frame}

\begin{frame}{填空}
	\linespread{1.2}
	\onslide<2->
	\begin{exampleblock}{{\bf 例1}\hfill}
		\begin{enumerate}
		  \item 曲线$\left\{\begin{array}{l}
		  	x=y^2\\ z=x^2
		  \end{array}\right.$在$(1,1,1)$处的法平面方程为
		  \underline{\;\uncover<3->{\alert{$2x+y+4z=7$}}\;}
		  \onslide<4->
		  \item 曲线$\left\{\begin{array}{l}
		  	xyz=2\\ x-y-z=0
		  \end{array}\right.$在$(2,1,1)$处的一个切向量与$z$轴
		  成锐角,则此向量与$y$轴正向的夹角为\underline{\;\uncover<5->
		  {\alert{$3\pi/4$}}\;}
		\end{enumerate}
	\end{exampleblock}
\end{frame}
% 
\begin{frame}
	\linespread{1.2}
	\begin{exampleblock}{{\bf 例1}\hfill}
		\begin{enumerate}
		  \addtocounter{enumi}{2}
		  \item 曲线$\left\{\begin{array}{l}
		  	y=2\sin x\\ z=3\cos x
		  \end{array}\right.$在$(0,0,3)$处的切线方程为
		  \underline{\;\uncover<2->{\alert{$\df{x}{-1}
		  =\df y2=\df{z-3}0$}}\;}
		  \onslide<3->
		  \bigskip
		  \item 曲面$z=4-x^2-y^2$在点\underline{\;\uncover<4->
		  {\alert{$(1,1,2)$}}\;}
		  处的切平面与平面$2x+2y+z=1$平行。
		\end{enumerate}
	\end{exampleblock}
\end{frame}

\begin{frame}{补充例题}
	\linespread{1.2}
	\begin{exampleblock}{{\bf 例2}\hfill}
		求函数$z=x^2-xy+y^2$在点$(1,1)$处沿与$x$轴正向成
		$\alpha$角的方向$\bm{l}$的方向导数。在怎样的方向上
		此导数取最大、最小及$0$值?
	\end{exampleblock}
	\bigskip\pause
	\begin{exampleblock}{{\bf 例3}\hfill}
		设$u=u(x,y,z)$,证明:$u$为$x,y,z$的线性函数,当且仅当
		$\bigtriangledown u$为常矢量。
	\end{exampleblock}
\end{frame}

\begin{frame}{补充例题}
	\linespread{1.2}
	\begin{exampleblock}{{\bf 例4}\hfill}
		试证明:当$x,y$很小时,有如下近似公式:
		$$(1+x)^m(1+y)^n\approx 1+mx+ny$$
	\end{exampleblock}
	\bigskip\pause
	\begin{exampleblock}{{\bf 例4'}\hfill}
		试证明:当$x,y$很小时,有如下近似公式:
		\begin{enumerate}
		  \item $\ln(1+x)\ln(1+y)\approx xy$
		  \item $\arctan\df{x+y}{1+xy}\approx x+y$
		\end{enumerate}
	\end{exampleblock}
\end{frame}

\begin{frame}
	\linespread{1.2}
	\begin{exampleblock}{{\bf 例5}\hfill}
		证明函数$f(x,y)=2x^2-3xy^2+y^4$不存在极值。
	\end{exampleblock}
	\pause 
	\bigskip
	\begin{exampleblock}{{\bf 例6}\hfill}
		证明函数$f(x,y)=4-ye^{\cos x}-y^2$在全平面上有无穷多个
		极大值,但无极小值。
	\end{exampleblock}
\end{frame}

\begin{frame}
	\linespread{1.2}
	\begin{exampleblock}{{\bf 例7}\hfill}
		在半径为$R$的圆的内接三角形中,求面积最大者。
	\end{exampleblock}
	\pause 
	\bigskip
% 	\begin{exampleblock}{{\bf 例8}\hfill}
% 		试证曲面$\sqrt{x}+\sqrt y+\sqrt z=\sqrt a\,(a>0)$
% 		在任一点处的切平面在三个坐标轴上的截距之和为常数。
% 	\end{exampleblock}
	\begin{exampleblock}{{\bf 例8}\hfill}
		现有资金$36$元,要建造一个无盖的长方体容器,已知底面造价为$3$元每平方米,
		侧面造价为$1$元每平方米,求可造出的长方体的最大容积。
	\end{exampleblock}
\end{frame}

% \begin{frame}
% 	\linespread{1.2}
% 	\begin{exampleblock}{{\bf 例2}\hfill}
% 		证明函数$f(x,y)=2x^2-3xy^2+y^4$不存在极值。
% 	\end{exampleblock}
% 	\pause 
% 	\begin{exampleblock}{{\bf 例3}\hfill}
% 		现有资金$36$元,要建造一个无盖的长方体容器,已知底面造价为$3$元每平方米,
% 		侧面造价为$1$元每平方米,求可造出的长方体的最大容积。
% 	\end{exampleblock}
% 	\pause 
% 	\begin{exampleblock}{{\bf 例4}\hfill}
% 		若生产容积为$4$立方米的长方体盒子,应如何设计,可使盒子的表面积最小。 
% 	\end{exampleblock}
% \end{frame}

\begin{frame}
	\linespread{1.2}
	\begin{exampleblock}{{\bf 例9}\hfill}
		当$x,y,z$均大于零时,求函数$u=\ln x+2\ln y+3\ln z$
		在球面$x^2+y^2+z^2=6R^2$上的最大值,并证明对任意正数
		$a,b,c$,成立不等式
		$$ab^2c^3\leq 108\left(\df{a+b+c}{6}\right)^6$$
	\end{exampleblock}
\end{frame}

\begin{frame}
	\linespread{1.2}
	{\small
		{\bf 解:}由Lagrange乘子法,记
		$$L=xy^2z^3+\lambda(x^2+y^2+z^2-6R^2),$$
		令$\bigtriangledown L=0$,可解得$x=R,y=\sqrt2 R,z=\sqrt3 R$,从而有:
		$$xy^2z^3\leq 6\sqrt 3R^6=6\sqrt 3\left(\df{x^2+y^2+z^2}{6}\right)^3.$$
		也即
		$$x^2y^4z^6\leq 108\left(\df{x^2+y^2+z^2}{6}\right)^6.$$
		记$a=x^2,b=y^2,c=z^2$,即证。
	}
\end{frame}

\begin{frame}
	\linespread{1.4}
	\begin{exampleblock}{{\bf 例10}\hfill}
		设$F(x,y,z)$在条件$\varphi(x,y,z)=0$和$\psi(x,y,z)=0$
		之下在点$P_0(x_0,y_0,z_0)$处取得极值$m$,证明:曲面
		$F(x,y,z)=m,\varphi(x,y,z)=0$和$\psi(x,y,z)=0$
		在$P_0$的法线共面,其中函数$F,\varphi,\psi$均有连续且不同时
		为零的一阶偏导数。
	\end{exampleblock}
\end{frame}

\begin{frame}
	\linespread{1.2}
	\begin{exampleblock}{{\bf 例11}\hfill}
		设$f(x,y)$一阶偏导数连续,且对任意实数$t$,有
		$$f(tx,ty)=tf(x,y)$$
		证明:曲面$z=f(x,y)$上任一点$M$处的切平面与向径平行。
	\end{exampleblock}
	\pause
	\bigskip
	\begin{exampleblock}{{\bf 例12}\hfill}
		试证曲面$\sqrt{x}+\sqrt y+\sqrt z=\sqrt a\,(a>0)$
		在任一点处的切平面在三个坐标轴上的截距之和为常数。
	\end{exampleblock}
\end{frame}

\begin{frame}{课堂练习}
	\linespread{1.2}
	\begin{exampleblock}{{\bf 例13}\hfill}
		求函数$z=x^2-xy+y^2$在区域$D:|x|+|y|\leq 1$
		上的最大和最小值。
	\end{exampleblock}
	\bigskip
	\begin{exampleblock}{{\bf 例14}\hfill}
		已知曲面方程$z=xe^{y/x}$,证明:曲面上任一点$M$处的法线与其
		向径垂直。
	\end{exampleblock}
\end{frame}

%=====================================
 
% \begin{frame}{title}
% 	\linespread{1.2}
% 	\begin{exampleblock}{{\bf title}\hfill}
% 		123
% 	\end{exampleblock}
% \end{frame}
% 
% \begin{frame}{title}
% 	\linespread{1.2}
% 	\begin{block}{{\bf title}\hfill}
% 		123
% 	\end{block}
% \end{frame}